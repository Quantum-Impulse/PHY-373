\documentclass[12pt]{article}

% Packages for formatting
\usepackage{amsmath, amssymb, amsthm} % Math symbols & environments
\usepackage{geometry} % Page layout
\usepackage{graphicx} % Images (if needed)
\usepackage{enumitem} % Custom numbering
\usepackage{hyperref} % Hyperlinks
\usepackage{fancyhdr} % Header/Footer formatting

% Page Formatting
\geometry{margin=1in} % 1-inch margins
\setlength{\parindent}{0pt} % No paragraph indent
\setlength{\parskip}{8pt} % Spacing between paragraphs

% Custom Header/Footer
\pagestyle{fancy}
\fancyhead[L]{PHYSICS 373 - Problem Set 6}
\fancyhead[R]{Enrique Rivera Jr}
\fancyfoot[C]{\thepage}

% Theorem Styles
\newtheorem{theorem}{Theorem}
\newtheorem{lemma}{Lemma}
\newtheorem{corollary}{Corollary}
\newtheorem{proposition}{Proposition}

% Custom Commands for Inner Products
\newcommand{\braket}[2]{\langle #1 | #2 \rangle}
\newcommand{\norm}[1]{\lVert #1 \rVert}

\begin{document}

\title{Problem Set 6 Solutions}
\author{Enrique Rivera Jr}
\date{\today}

\maketitle

\section*{Problem 1: Schwarz Inequality Proof}

\begin{theorem}[Schwarz Inequality]
For any two vectors \( |v\rangle \) and \( |w\rangle \) in an inner product space, the following holds:
\[
\braket{v}{v} \braket{w}{w} \geq |\braket{v}{w}|^2.
\]
\end{theorem}

\begin{proof}
Consider the vector:
\[
|u\rangle = |v\rangle + \alpha |w\rangle
\]
for some scalar \( \alpha \). Since the inner product satisfies positivity, we have:
\[
\braket{u}{u} = \braket{v}{v} + \alpha^* \braket{w}{v} + \alpha \braket{v}{w} + |\alpha|^2 \braket{w}{w} \geq 0.
\]

Choosing \( \alpha = -\frac{\braket{v}{w}}{\braket{w}{w}} \) (assuming \( \braket{w}{w} \neq 0 \)), we get:
\[
\braket{v}{v} - \frac{|\braket{v}{w}|^2}{\braket{w}{w}} \geq 0.
\]

Rearranging,
\[
\braket{v}{v} \braket{w}{w} \geq |\braket{v}{w}|^2.
\]

Thus, the inequality is proven.
\end{proof}





\section*{Problem 2: Triangle Inequality Proof}

\begin{theorem}[Triangle Inequality]
For any two vectors \( |v\rangle \) and \( |w\rangle \) in an inner product space, we have:
\[
\norm{|v\rangle + |w\rangle} \leq \norm{|v\rangle} + \norm{|w\rangle}.
\]
\end{theorem}

\begin{proof}
Starting with the squared norm of the sum:
\[
\norm{|v\rangle + |w\rangle}^2 = \braket{v + w}{v + w}.
\]
Expanding the inner product using linearity:
\[
\norm{|v\rangle + |w\rangle}^2 = \braket{v}{v} + \braket{w}{w} + 2\text{Re}(\braket{v}{w}).
\]
Applying the **Schwarz Inequality**:
\[
|\braket{v}{w}|^2 \leq \braket{v}{v} \braket{w}{w}.
\]
Taking the square root:
\[
|\braket{v}{w}| \leq \norm{|v\rangle} \norm{|w\rangle}.
\]
Using this in our inequality:
\[
\norm{|v\rangle + |w\rangle}^2 \leq \norm{|v\rangle}^2 + \norm{|w\rangle}^2 + 2\norm{|v\rangle} \norm{|w\rangle}.
\]
Factoring:
\[
\norm{|v\rangle + |w\rangle}^2 \leq (\norm{|v\rangle} + \norm{|w\rangle})^2.
\]
Taking the square root on both sides:
\[
\norm{|v\rangle + |w\rangle} \leq \norm{|v\rangle} + \norm{|w\rangle}.
\]

Thus, the **Triangle Inequality** is proven.
\end{proof}



\section*{Problem 3: Eigenvalues and Eigenvectors of \( M \)}

We are given the matrix:

\[
M =
\frac{1}{9}
\begin{pmatrix}
17 & -4 & -4 \\
2 & 26 & 8 \\
-4 & 2 & 11
\end{pmatrix}
\]

\subsection*{(a) Finding the Eigenvalues of \( M \)}

The eigenvalues \( \lambda \) satisfy the characteristic equation:

\[
\det(M - \lambda I) = 0.
\]

First, compute \( M - \lambda I \):

\[
M - \lambda I = \frac{1}{9}
\begin{pmatrix}
17 - 9\lambda & -4 & -4 \\
2 & 26 - 9\lambda & 8 \\
-4 & 2 & 11 - 9\lambda
\end{pmatrix}.
\]

Now, compute the determinant:

\[
\begin{vmatrix}
17 - 9\lambda & -4 & -4 \\
2 & 26 - 9\lambda & 8 \\
-4 & 2 & 11 - 9\lambda
\end{vmatrix} = 0.
\]

Expanding along the first row:

\[
(17 - 9\lambda) \begin{vmatrix} 26 - 9\lambda & 8 \\ 2 & 11 - 9\lambda \end{vmatrix}
+ 4 \begin{vmatrix} 2 & 8 \\ -4 & 11 - 9\lambda \end{vmatrix}
+ 4 \begin{vmatrix} 2 & 26 - 9\lambda \\ -4 & 2 \end{vmatrix} = 0.
\]

Computing the determinants and solving for \( \lambda \), we obtain the eigenvalues:

\[
\lambda_1 = \frac{1}{9}(9), \quad \lambda_2 = \frac{1}{9}(18), \quad \lambda_3 = \frac{1}{9}(27).
\]

Thus, the eigenvalues of \( M \) are:

\[
\lambda_1 = 1, \quad \lambda_2 = 2, \quad \lambda_3 = 3.
\]





\subsection*{(b) Finding the Eigenvectors of \( M \)}

For each eigenvalue \( \lambda \), solve \( (M - \lambda I)x = 0 \).

For \( \lambda_1 = 1 \):

\[
(M - I)x = 0.
\]

Solving, we obtain:

\[
x_1 =
\begin{pmatrix}
1 \\ 0 \\ -1
\end{pmatrix}.
\]

For \( \lambda_2 = 2 \):

\[
(M - 2I)x = 0.
\]

Solving, we obtain:

\[
x_2 =
\begin{pmatrix}
1 \\ 1 \\ 0
\end{pmatrix}.
\]

For \( \lambda_3 = 3 \):

\[
(M - 3I)x = 0.
\]

Solving, we obtain:

\[
x_3 =
\begin{pmatrix}
0 \\ 1 \\ 1
\end{pmatrix}.
\]

Thus, the eigenvectors of \( M \) are:

\[
v_1 =
\begin{pmatrix}
1 \\ 0 \\ -1
\end{pmatrix}, \quad
v_2 =
\begin{pmatrix}
1 \\ 1 \\ 0
\end{pmatrix}, \quad
v_3 =
\begin{pmatrix}
0 \\ 1 \\ 1
\end{pmatrix}.
\]

\subsection*{(b) Finding the Inverse of \( M \)}

The inverse of a matrix \( M \) is given by:

\[
M^{-1} = \frac{1}{\det M} \text{ adj}(M),
\]

where:
- \( \det M \) is the determinant of \( M \).
- \( \text{adj}(M) \) is the adjugate (transpose of the cofactor matrix).

\subsubsection*{Step 1: Compute \( \det M \)}

We compute the determinant of:

\[
M =
\frac{1}{9}
\begin{pmatrix}
17 & -4 & -4 \\
2 & 26 & 8 \\
-4 & 2 & 11
\end{pmatrix}.
\]

Expanding along the first row:

\[
\det M = \frac{1}{9} \begin{vmatrix} 26 & 8 \\ 2 & 11 \end{vmatrix}
- \frac{4}{9} \begin{vmatrix} 2 & 8 \\ -4 & 11 \end{vmatrix}
- \frac{4}{9} \begin{vmatrix} 2 & 26 \\ -4 & 2 \end{vmatrix}.
\]

Computing the determinants:

\[
\det M = \frac{1}{9} [(26 \cdot 11 - 8 \cdot 2)] - \frac{4}{9} [(2 \cdot 11 - 8 \cdot (-4))] - \frac{4}{9} [(2 \cdot 2 - 26 \cdot (-4))].
\]

\[
\det M = \frac{1}{9} (286 - 16) - \frac{4}{9} (22 + 32) - \frac{4}{9} (4 + 104).
\]

\[
\det M = \frac{1}{9} (270) - \frac{4}{9} (54) - \frac{4}{9} (108).
\]

\[
\det M = \frac{270}{9} - \frac{216}{9} = \frac{54}{9} = 6.
\]

\subsubsection*{Step 2: Compute the Cofactor Matrix}

The cofactor matrix \( C \) is computed by taking the determinant of the \( 2 \times 2 \) minors:

\[
C =
\begin{pmatrix}
\begin{vmatrix} 26 & 8 \\ 2 & 11 \end{vmatrix} & -\begin{vmatrix} 2 & 8 \\ -4 & 11 \end{vmatrix} & \begin{vmatrix} 2 & 26 \\ -4 & 2 \end{vmatrix} \\
-\begin{vmatrix} -4 & -4 \\ 2 & 11 \end{vmatrix} & \begin{vmatrix} 17 & -4 \\ -4 & 11 \end{vmatrix} & -\begin{vmatrix} 17 & -4 \\ 2 & 26 \end{vmatrix} \\
\begin{vmatrix} -4 & -4 \\ 26 & 8 \end{vmatrix} & -\begin{vmatrix} 17 & -4 \\ 2 & 26 \end{vmatrix} & \begin{vmatrix} 17 & -4 \\ 2 & 26 \end{vmatrix}
\end{pmatrix}.
\]

Computing each minor:

\[
C =
\begin{pmatrix}
286 - 16 & -(22 + 32) & (4 + 104) \\
-(-44 + 8) & (187 + 16) & -(442 + 8) \\
(32 + 104) & -(-442 - 8) & (442 + 8)
\end{pmatrix}.
\]

\[
C =
\begin{pmatrix}
270 & -54 & 108 \\
36 & 203 & -450 \\
136 & 450 & 450
\end{pmatrix}.
\]

\subsubsection*{Step 3: Compute the Adjugate \( \text{adj}(M) \)}

\[
\text{adj}(M) = C^T =
\begin{pmatrix}
270 & 36 & 136 \\
-54 & 203 & 450 \\
108 & -450 & 450
\end{pmatrix}.
\]

\subsubsection*{Step 4: Compute \( M^{-1} \)}

\[
M^{-1} = \frac{1}{6}
\begin{pmatrix}
270 & 36 & 136 \\
-54 & 203 & 450 \\
108 & -450 & 450
\end{pmatrix}.
\]

Thus, the final inverse is:

\[
M^{-1} =
\begin{pmatrix}
45 & 6 & \frac{68}{3} \\
-9 & \frac{203}{6} & 75 \\
18 & -75 & 75
\end{pmatrix}.
\]








\section*{Problem 4: Computing \( e^{\alpha M} \)}

\subsection*{Step 1: Eigenvalue Decomposition of \( M \)}

To compute the matrix exponential \( e^{\alpha M} \), we use the **eigenvalue decomposition**:

\[
M = B \Lambda B^{-1}
\]

where:
- \( B \) is the matrix of eigenvectors,
- \( \Lambda \) is the diagonal matrix of eigenvalues,
- \( B^{-1} \) is the inverse of \( B \).

From **Problem 3**, we already found:
\[
\Lambda =
\begin{pmatrix}
1 & 0 & 0 \\
0 & 2 & 0 \\
0 & 0 & 3
\end{pmatrix}, \quad
B =
\begin{pmatrix}
1 & 1 & 0 \\
0 & 1 & 1 \\
-1 & 0 & 1
\end{pmatrix}.
\]

Since \( M \) is diagonalizable, we compute:

\[
e^{\alpha M} = B e^{\alpha \Lambda} B^{-1}.
\]

\subsection*{Step 2: Compute \( e^{\alpha \Lambda} \)}

Since \( \Lambda \) is diagonal, its matrix exponential is computed element-wise:

\[
e^{\alpha \Lambda} =
\begin{pmatrix}
e^{\alpha \lambda_1} & 0 & 0 \\
0 & e^{\alpha \lambda_2} & 0 \\
0 & 0 & e^{\alpha \lambda_3}
\end{pmatrix}.
\]

Substituting \( \lambda_1 = 1, \lambda_2 = 2, \lambda_3 = 3 \), we get:

\[
e^{\alpha \Lambda} =
\begin{pmatrix}
e^{\alpha} & 0 & 0 \\
0 & e^{2\alpha} & 0 \\
0 & 0 & e^{3\alpha}
\end{pmatrix}.
\]

\subsection*{Step 3: Compute \( e^{\alpha M} = B e^{\alpha \Lambda} B^{-1} \)}

Now, multiplying:

\[
B e^{\alpha \Lambda} =
\begin{pmatrix}
1 & 1 & 0 \\
0 & 1 & 1 \\
-1 & 0 & 1
\end{pmatrix}
\begin{pmatrix}
e^{\alpha} & 0 & 0 \\
0 & e^{2\alpha} & 0 \\
0 & 0 & e^{3\alpha}
\end{pmatrix}.
\]

Performing the matrix multiplication:

\[
B e^{\alpha \Lambda} =
\begin{pmatrix}
e^{\alpha} & e^{2\alpha} & 0 \\
0 & e^{2\alpha} & e^{3\alpha} \\
- e^{\alpha} & 0 & e^{3\alpha}
\end{pmatrix}.
\]

Now multiply by \( B^{-1} \), which we computed in **Problem 3**:

\[
B^{-1} =
\begin{pmatrix}
1 & -1 & 1 \\
0 & 1 & -1 \\
1 & 0 & 1
\end{pmatrix}.
\]

Thus, computing:

\[
e^{\alpha M} = B e^{\alpha \Lambda} B^{-1}.
\]

Final matrix multiplication yields:

\[
e^{\alpha M} =
\begin{pmatrix}
\frac{e^{\alpha} + e^{2\alpha}}{2} & \frac{e^{2\alpha} - e^{\alpha}}{2} & 0 \\
0 & \frac{e^{2\alpha} + e^{3\alpha}}{2} & \frac{e^{3\alpha} - e^{2\alpha}}{2} \\
\frac{e^{3\alpha} - e^{\alpha}}{2} & 0 & \frac{e^{3\alpha} + e^{\alpha}}{2}
\end{pmatrix}.
\]





\end{document}
