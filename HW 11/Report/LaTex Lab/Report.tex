\documentclass[12pt]{article}

% Packages for formatting
\usepackage{amsmath, amssymb, amsthm} % Math symbols & environments
\usepackage{geometry} % Page layout
\usepackage{graphicx} % Images (if needed)
\usepackage{enumitem} % Custom numbering
\usepackage{hyperref} % Hyperlinks
\usepackage{fancyhdr} % Header/Footer formatting
\usepackage{physics} % Includes \ket and \bra notation

% Page Formatting
\geometry{margin=1in} % 1-inch margins
\setlength{\parindent}{0pt} % No paragraph indent
\setlength{\parskip}{8pt} % Spacing between paragraphs

% Custom Header/Footer
\pagestyle{fancy}
\fancyhead[L]{PHYSICS 373 - Problem Set 10}
\fancyhead[R]{Enrique Rivera Jr}
\fancyfoot[C]{\thepage}

% Theorem Styles
\newtheorem{theorem}{Theorem}
\newtheorem{lemma}{Lemma}
\newtheorem{corollary}{Corollary}
\newtheorem{proposition}{Proposition}

\begin{document}

\title{Problem Set 10 Solutions}
\author{Enrique Rivera Jr}
\date{\today}

\maketitle

% ------------------------------------------------------------------------------------------------
% Question 1
% ------------------------------------------------------------------------------------------------


\section*{Question 1: Expressing $|2; \ell,m\rangle$ in the $|n_1,n_2,n_3\rangle$ Basis}

We have two ways to label the $n=2$ energy eigenstates of the 3D isotropic harmonic oscillator:
\begin{itemize}
  \item \textbf{Cartesian basis:} $|n_1,n_2,n_3\rangle$, where $n_1+n_2+n_3=2$. There are exactly 6 such states:
  \begin{equation*}
    |2,0,0\rangle,\;|0,2,0\rangle,\;|0,0,2\rangle,\;|1,1,0\rangle,\;|1,0,1\rangle,\;|0,1,1\rangle.
  \end{equation*}
  \item \textbf{Spherical basis:} $|n;\ell,m\rangle$ with $n=2$. Then $n=2$ can split into $(n_r,\ell)$ satisfying $2n_r+\ell=2$. We get:
  \begin{itemize}
    \item $\ell=0,\,n_r=1$ (1 state: $|2;0,0\rangle$),
    \item $\ell=2,\,n_r=0$ (5 states: $|2;2,m\rangle$, $m=-2\dots+2$).
  \end{itemize}
  So total of 6 states again.
\end{itemize}

Below we show how each $|2;\ell,m\rangle$ is written as a linear combination of $|n_1,n_2,n_3\rangle$.

\subsection*{1. The $\ell=2$ Multiplet (Five States)}
Since $\ell=2$, $m$ runs from $-2$ to $+2$. We have 5 states: $m=-2,-1,0,1,2$.

\paragraph{(a) $|2;2,\pm2\rangle$.}
It is given (and can be verified by symmetry arguments) that:
\[
\boxed{
|2;2,\pm 2\rangle
= \frac{1}{2}\Bigl[|2,0,0\rangle \;-\; |0,2,0\rangle \;\pm\; i\,\sqrt{2}\,|1,1,0\rangle\Bigr].
}
\]
The $\pm2$ indicates that we attach the $\pm i$ factor times $\sqrt{2}$. This ensures these states behave like the $m=\pm2$ spherical harmonics.

\paragraph{(b) $|2;2,0\rangle$.}
One finds
\[
\boxed{
|2;2,0\rangle = \frac{1}{\sqrt{6}}\Bigl[|2,0,0\rangle + |0,2,0\rangle \;-\; 2\,|0,0,2\rangle\Bigr].
}
\]
Here we see a combination that is symmetric in $(x,y)$ but subtracts out the $z$ direction in a certain proportion. This matches the $\ell=2,m=0$ spherical harmonic in the HO Fock basis.

\paragraph{(c) $|2;2,\pm1\rangle$.}
Often written as linear combinations involving $|1,0,1\rangle$ and $|0,1,1\rangle$, plus some portion of $|2,0,0\rangle$ and $|0,2,0\rangle$, with appropriate phases. A typical set (with some sign conventions) is:
\begin{align*}
|2;2,+1\rangle &= \frac{1}{2}\Bigl[|2,0,0\rangle + |0,2,0\rangle \;-\;\sqrt{2}\,(|1,0,1\rangle + i\,|0,1,1\rangle)\Bigr],\\
|2;2,-1\rangle &= \frac{1}{2}\Bigl[|2,0,0\rangle + |0,2,0\rangle \;+\;\sqrt{2}\,(|1,0,1\rangle - i\,|0,1,1\rangle)\Bigr].
\end{align*}
Exact sign details can vary based on phase conventions.

\subsection*{2. The $\ell=0$ State (One State)}
For $n=2$ and $\ell=0$, we must have $n_r=1$. The single state is $|2;0,0\rangle$. By orthogonality with the $\ell=2$ subspace, we find:
\[
\boxed{
|2;0,0\rangle
= \frac{1}{\sqrt{3}}\Bigl[|1,1,0\rangle + |1,0,1\rangle + |0,1,1\rangle\Bigr].
}
\]
That combination is fully symmetric among the $(x,y,z)$ directions and carries no angular momentum ($\ell=0$).

\subsection*{Conclusion}
These six states $|2;\ell,m\rangle$ in spherical coordinates map to linear combinations of the six $|n_1,n_2,n_3\rangle$ with $n_1+n_2+n_3=2$. As examples:
\begin{itemize}
\item $|2;2,\pm2\rangle = \tfrac{1}{2}\bigl(|2,0,0\rangle - |0,2,0\rangle \pm i\sqrt{2}\,|1,1,0\rangle\bigr)$,
\item $|2;0,0\rangle = \tfrac{1}{\sqrt{3}}\bigl(|1,1,0\rangle + |1,0,1\rangle + |0,1,1\rangle\bigr)$,
\item $|2;2,\pm1\rangle$ have slightly more complicated combos involving $|1,0,1\rangle$ and $|0,1,1\rangle$ as well.
\end{itemize}
These superpositions exhaust the full $n=2$ manifold in both bases.



% ------------------------------------------------------------------------------------------------
% Question 2
% ------------------------------------------------------------------------------------------------


\section*{Question 2: $\Delta x$ and $\Delta p_x$ in the Hydrogen Ground State}

We consider the hydrogenic ground state, $\psi_{100}(r)$, given by:
\begin{equation}
\psi_{100}(r) = \frac{1}{\sqrt{\pi a_0^3}}\, e^{-r/a_0},
\end{equation}
where $a_0$ is the Bohr radius. We want $\Delta x$ and $\Delta p_x$.

\subsection*{1. $\Delta x$}
Since the ground state is spherically symmetric, $\langle x \rangle = 0$, and $\langle x^2\rangle = \frac{1}{3}\langle r^2\rangle$. A standard result (or from integrals) is $\langle r^2\rangle = 3 a_0^2$ for $n=1$ hydrogen. Hence:
\begin{align*}
\langle x^2 \rangle &= \frac{1}{3}\times 3 a_0^2 = a_0^2,\\
\Delta x &= \sqrt{\langle x^2\rangle - \langle x\rangle^2} = a_0.
\end{align*}

\subsection*{2. $\Delta p_x$}
In momentum space, one can compute the Fourier transform of $\psi_{100}$, or use that $\langle \mathbf{p}^2\rangle = \hbar^2 / a_0^2$ in the hydrogen ground state. Again by symmetry, $\langle p_x^2\rangle = \frac{1}{3}\langle \mathbf{p}^2\rangle = \frac{\hbar^2}{3 a_0^2}$. Thus:
\begin{align*}
\Delta p_x &= \sqrt{\langle p_x^2\rangle} = \sqrt{\frac{\hbar^2}{3 a_0^2}} = \frac{\hbar}{\sqrt{3}\,a_0}.
\end{align*}

\subsection*{Final Results}
\begin{equation}
\boxed{
\Delta x = a_0, \quad\Delta p_x = \frac{\hbar}{\sqrt{3}\,a_0}.
}
\end{equation}


% ------------------------------------------------------------------------------------------------
% Question 3
% ------------------------------------------------------------------------------------------------
\section*{Question 3(a): Probability Inside Radius $b$ for $\psi_{2,1,m}$}

\subsectiThe hydrogenic wavefunction $\psi_{n,\ell,m}(r,\theta,\phi)$ factors as $R_{n,\ell}(r)\,Y_{\ell,m}(\theta,\phi)$. The probability inside radius $b$ is:
\begin{equation}
P(b) = \int_0^b \int_{\Omega} |\psi_{2,1,m}(r,\theta,\phi)|^2\,r^2\sin\theta\,dr\,d\theta\,d\phi.
\end{equation}
Because $\int_{\Omega} |Y_{1,m}|^2 d\Omega = 1$ for any $m$, we get:
\begin{equation}
P(b) = \int_0^b |R_{2,1}(r)|^2\,r^2\,dr.
\end{equation}
Hence the probability is independent of $m$. Also, as $b\to\infty$, $P\to1$.

\subsection*{2. Express in Terms of $b/a$}
The radial function for $n=2,\ell=1$ is typically
\begin{equation}
R_{2,1}(r) = \frac{1}{2\sqrt{6}\,a^{3/2}}\,\Bigl(\frac{r}{a}\Bigr) e^{-\frac{r}{2a}} \quad (a=\text{Bohr radius}).
\end{equation}
Thus $|R_{2,1}(r)|^2$ becomes a function of $r/a$. Defining $x=b/a$, we get
\begin{equation}
P(x) = \int_0^x F(\rho)\,d\rho,\quad \rho = r/a.
\end{equation}
This is cleaon*{1. Radial-Angular Factorization}
rly a function of $b/a$ alone, going to 1 as $b/a\to\infty$.

\subsection*{Answer (3a)}
\textbf{Probability inside radius $b$ is:}
\begin{equation}
P\bigl(\tfrac{b}{a}\bigr) = \int_0^{b} |R_{2,1}(r)|^2\,r^2\,dr,
\end{equation}
independent of $m$. Expressing $r$ in units of $a$ yields a dimensionless function of $b/a$, which $\to1$ as $b/a\to\infty$.







\section*{Question 3(b): Small-$\frac{b}{a}$ Expansion of Probability \& Finding $c$}

Recall from part (a), the probability inside radius $b$ for the state $\psi_{2,1,m}$ is:
\begin{equation}
P(b) = \int_0^b |R_{2,1}(r)|^2 \, r^2 \,dr,
\end{equation}
where
\begin{align*}
R_{2,1}(r) &= \frac{1}{2\sqrt{6}\,a^{3/2}}\,\Bigl(\frac{r}{a}\Bigr) e^{-r/(2a)},\\
|R_{2,1}(r)|^2 &= \frac{1}{24\,a^3}\,\Bigl(\frac{r}{a}\Bigr)^2 e^{-r/a}.
\end{align*}

\subsection*{1. Leading Behavior for $r \ll a$}
When $r\ll a$, $e^{-r/a}\approx 1$. Hence
\begin{equation}
|R_{2,1}(r)|^2 \approx \frac{1}{24\,a^3}\,\Bigl(\frac{r}{a}\Bigr)^2
= \frac{r^2}{24\,a^5}.
\end{equation}
Thus,
\begin{align*}
P(b) &= \int_0^b |R_{2,1}(r)|^2 \, r^2\,dr \approx \int_0^b \frac{r^2}{24\,a^5}\,r^2\,dr
= \frac{1}{24\,a^5} \int_0^b r^4\,dr \\[6pt]
&= \frac{1}{24\,a^5}\,\frac{b^5}{5} = \frac{b^5}{120\,a^5}.
\end{align*}

\subsection*{2. Hence $P(b) \sim c\,(\tfrac{b}{a})^5$}
We see that
\begin{equation}
P(b) \approx \frac{1}{120}\,\Bigl(\frac{b}{a}\Bigr)^5,
\end{equation}
so $c=\tfrac{1}{120}$. This confirms the desired form,
\begin{equation}
P\bigl(\tfrac{b}{a}\bigr) \sim \frac{1}{120}\,\Bigl(\tfrac{b}{a}\Bigr)^5\quad\text{for}\; b\ll a.
\end{equation}

\subsection*{Answer (3b)}
For small $b/a$, the probability $P(b)$ behaves like $c (b/a)^5$, with $c = \tfrac{1}{120}$.\newline



% ------------------------------------------------------------------------------------------------
% Question 4
% ------------------------------------------------------------------------------------------------
\section*{Question 4: Time Evolution in a Uniform Magnetic Field}
We have the Hamiltonian:
\begin{equation}
H = \frac{\hat{p}^2}{2m_e} - \frac{e^2}{4\pi \epsilon_0 \\,r} + \frac{eB}{2m_e}\,\hat{L}_z,
\end{equation}
and the initial wavefunction
\begin{equation}
\Psi(\mathbf{x},0) = \frac{1}{\sqrt{32\pi a^3}}\,\frac{r}{a}\,e^{-r/2a}\,\sin\theta\,\cos\phi.
\end{equation}
Below we find $\Psi(\mathbf{x},t)$.

\subsection*{1. Observing the $\hat{L}_z$ Term}
\begin{itemize}
\item The usual hydrogen Hamiltonian is $H_0 = \tfrac{\hat{p}^2}{2m_e} - \tfrac{e^2}{4\pi\epsilon_0}\,\tfrac{1}{r}$. Here we add $\tfrac{eB}{2m_e}\hat{L}_z$, so the energy eigenstates remain $|n,\ell,m\rangle$ but each acquires an energy shift $\Delta E = (eB/2m_e)(m\hbar)$.
\item Hence the full Hamiltonian is $H_0 + (\tfrac{eB}{2m_e}\hat{L}_z)$, and $|n,\ell,m\rangle$ are still eigenstates, with new energies:
\begin{equation}
E_{n,\ell,m} = E_{n,\ell}^{(0)} + \frac{eB}{2m_e}\,m\hbar.
\end{equation}
\end{itemize}

\subsection*{2. Identifying the Initial State}
We have
\begin{equation}
\Psi(\mathbf{x},0) = \Bigl(\text{radial factor}\Bigr)\,\sin\theta\,\cos\phi.
\end{equation}
One sees the radial factor $\propto r e^{-r/2a}$ suggests $n=2,\ell=1$. Meanwhile, $\sin\theta\,\cos\phi$ is a combination of $Y_{1,\pm1}(\theta,\phi)$, since $\cos\phi = \tfrac{1}{2}(e^{i\phi}+e^{-i\phi})$. So we get roughly an equal superposition of $m=+1$ and $m=-1$ with no $m=0$ component.

So (up to normalization checks)
\begin{equation}
\Psi(\mathbf{x},0) = \frac{1}{\sqrt{2}}\bigl(\psi_{2,1,+1} + \psi_{2,1,-1}\bigr).
\end{equation}

\subsection*{3. Time Evolution}
Since $\psi_{n,\ell,m}$ remain eigenstates of $H$, we simply attach phase factors $e^{-i E_{n,\ell,m} t/\hbar}$. Concretely:
\begin{align*}
E_{2,1,+1} &= E_{2,1}^{(0)} + \frac{eB}{2m_e}(+1)\hbar,\\
E_{2,1,-1} &= E_{2,1}^{(0)} + \frac{eB}{2m_e}(-1)\hbar.
\end{align*}
Hence:
\begin{align*}
\Psi(\mathbf{x},t)
&= \frac{1}{\sqrt{2}}\Bigl[e^{-i E_{2,1,+1} t/\hbar}\,\psi_{2,1,+1}(\mathbf{x}) + e^{-i E_{2,1,-1} t/\hbar}\,\psi_{2,1,-1}(\mathbf{x})\Bigr].
\end{align*}

\subsection*{Answer (Question 4)}
\begin{equation}
\boxed{
\Psi(\mathbf{x},t)
= \frac{1}{\sqrt{2}}\Bigl(e^{-\frac{i}{\hbar}E_{2,1,+1}t}\,\psi_{2,1,+1}(\mathbf{x}) \;+\; e^{-\frac{i}{\hbar}E_{2,1,-1}t}\,\psi_{2,1,-1}(\mathbf{x})\Bigr).
}
\end{equation}
where $E_{2,1,\pm1} = E_{2,1}^{(0)} + \tfrac{eB}{2m_e}(\pm1)\hbar.$






\end{document}
