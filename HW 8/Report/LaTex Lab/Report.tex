\documentclass[12pt]{article}

% Packages for formatting
\usepackage{amsmath, amssymb, amsthm} % Math symbols & environments
\usepackage{geometry} % Page layout
\usepackage{graphicx} % Images (if needed)
\usepackage{enumitem} % Custom numbering
\usepackage{hyperref} % Hyperlinks
\usepackage{fancyhdr} % Header/Footer formatting
\usepackage{physics} % Includes \ket and \bra notation

% Page Formatting
\geometry{margin=1in} % 1-inch margins
\setlength{\parindent}{0pt} % No paragraph indent
\setlength{\parskip}{8pt} % Spacing between paragraphs

% Custom Header/Footer
\pagestyle{fancy}
\fancyhead[L]{PHYSICS 373 - Problem Set 8}
\fancyhead[R]{Enrique Rivera Jr}
\fancyfoot[C]{\thepage}

% Theorem Styles
\newtheorem{theorem}{Theorem}
\newtheorem{lemma}{Lemma}
\newtheorem{corollary}{Corollary}
\newtheorem{proposition}{Proposition}

\begin{document}

\title{Problem Set 8 Solutions}
\author{Enrique Rivera Jr}
\date{\today}

\maketitle





\section*{Question 1(a): Ground State and First Excited State of the 3D Isotropic Harmonic Oscillator}
\subsection*{Hamiltonian and Separation of Variables}
\noindent
The Hamiltonian for the 3D isotropic harmonic oscillator is:
\begin{equation}
\hat{H} = \frac{\hat{p}_x^2 + \hat{p}_y^2 + \hat{p}_z^2}{2m} + \frac{1}{2}m\omega^2 (x^2 + y^2 + z^2).
\end{equation}
Because it \emph{separates} into a sum of three identical 1D harmonic oscillators, the wavefunction can be written as a product:
\begin{equation}
\Psi_{n_x,n_y,n_z}(x,y,z) = \psi_{n_x}(x)\,\psi_{n_y}(y)\,\psi_{n_z}(z),
\end{equation}
where each \(\psi_{n}(x)\) is the standard 1D harmonic oscillator solution.

\subsection*{1D Review}
\noindent\emph{1D ground state}:\quad
\begin{equation}
\psi_{0}(x) = \biggl(\frac{m\omega}{\pi\hbar}\biggr)^{\!1/4}\exp\bigl[-\tfrac{m\omega}{2\hbar}\,x^2\bigr].
\end{equation}
\noindent\emph{1D first excited state}:\quad
\begin{equation}
\psi_{1}(x) = \biggl(\frac{m\omega}{\pi\hbar}\biggr)^{\!1/4}\,\sqrt{2}\,\alpha\,x\,\exp\bigl[-\tfrac{m\omega}{2\hbar}\,x^2\bigr]\quad\text{where}\quad\alpha = \Bigl(\frac{m\omega}{\hbar}\Bigr)^{1/2}.
\end{equation}

\subsection*{Ground State in 3D}
\noindent
For \(n_x=n_y=n_z=0\), the 3D ground state is:
\begin{equation}
\Psi_{0,0,0}(x,y,z) = \psi_0(x)\,\psi_0(y)\,\psi_0(z)
= \biggl(\frac{m\omega}{\pi\hbar}\biggr)^{3/4}\exp\Bigl[-\frac{m\omega}{2\hbar}(x^2 + y^2 + z^2)\Bigr].
\end{equation}

\subsection*{First Excited States in 3D}
\noindent
The energy in 3D is \(\hbar\omega\,(n_x + n_y + n_z + \tfrac32)\). The first excited level occurs when \(n_x + n_y + n_z = 1\). That yields three possible combinations:
\begin{align*}
(n_x,n_y,n_z)&=(1,0,0),\\
(n_x,n_y,n_z)&=(0,1,0),\\
(n_x,n_y,n_z)&=(0,0,1).
\end{align*}
Thus we have three degenerate first-excited states:
\begin{align*}
\Psi_{1,0,0}(x,y,z) &= \psi_1(x)\,\psi_0(y)\,\psi_0(z),\\
\Psi_{0,1,0}(x,y,z) &= \psi_0(x)\,\psi_1(y)\,\psi_0(z),\\
\Psi_{0,0,1}(x,y,z) &= \psi_0(x)\,\psi_0(y)\,\psi_1(z),
\end{align*}
all with energy \(E = \tfrac{5}{2}\hbar\omega\).

\subsection*{Final Results}
\begin{itemize}
\item \textbf{Ground State:}
\begin{equation}
\Psi_{0,0,0}(x,y,z) = \biggl(\frac{m\omega}{\pi\hbar}\biggr)^{3/4}\exp\Bigl[-\frac{m\omega}{2\hbar}(x^2 + y^2 + z^2)\Bigr].
\end{equation}
\item \textbf{First-Excited States (3-fold degenerate):}
\begin{align*}
\Psi_{1,0,0}(x,y,z) &= \psi_1(x)\,\psi_0(y)\,\psi_0(z),\\
\Psi_{0,1,0}(x,y,z) &= \psi_0(x)\,\psi_1(y)\,\psi_0(z),\\
\Psi_{0,0,1}(x,y,z) &= \psi_0(x)\,\psi_0(y)\,\psi_1(z),
\end{align*}
\item \textbf{Energy at first-excited level:}\quad \(E = \frac{5}{2}\hbar\omega.\)
\end{itemize}





\section*{Question 1(b): Energy and Degeneracy of the 3D Isotropic Harmonic Oscillator}

\subsection*{Energy of the n-th Excited State}
For a 1D harmonic oscillator, the energy levels are
\begin{equation}
E_n^{(\text{1D})} = \hbar \omega \Bigl(n + \tfrac12\Bigr), \quad n = 0,1,2,\dots
\end{equation}

In 3D, we have three independent 1D oscillators, so the total energy is
\begin{equation}
E_{n_x,n_y,n_z} = \hbar\omega\Bigl(n_x + n_y + n_z + \tfrac32\Bigr).
\end{equation}
Defining \(n = n_x + n_y + n_z\), the \(n\)-th excited level has energy
\begin{equation}
\boxed{E_n = \hbar \omega \bigl(n + \tfrac{3}{2}\bigr).}
\end{equation}

\subsection*{Degeneracy of the n-th Excited State}
The states with \(n_x + n_y + n_z = n\) share the same energy \(E_n\). The number of nonnegative integer solutions to \(n_x + n_y + n_z = n\) is
\begin{equation}
\binom{n + 3 - 1}{3 - 1} = \binom{n + 2}{2} = \frac{(n+1)(n+2)}{2}.
\end{equation}
Hence, the degeneracy is
\begin{equation}
\boxed{g_n = \frac{(n+1)(n+2)}{2}.}
\end{equation}

\subsection*{Construction Using Raising Operators}
Let \(a_i, a_i^\dagger\) be the annihilation and creation operators for each Cartesian direction \(i = 1,2,3\). The 3D ground state \(\ket{\psi_0}\) satisfies
\begin{equation}
a_i \ket{\psi_0} = 0, \quad i = 1,2,3.
\end{equation}
Then any excited state with quantum numbers \((n_x, n_y, n_z)\) can be built by applying the appropriate creation operators:
\begin{equation}
\ket{n_x, n_y, n_z} \;\propto\; (a_1^\dagger)^{n_x}\,(a_2^\dagger)^{n_y}\,(a_3^\dagger)^{n_z}\,\ket{\psi_0}.
\end{equation}
Hence, the \(n\)-th excited state (\(n = n_x + n_y + n_z\)) can be written in the form
\begin{equation}
\boxed{(a_1^\dagger)^{n_1}(a_2^\dagger)^{n_2}(a_3^\dagger)^{n_3} \ket{\psi_0} \quad\text{with}\quad n_1 + n_2 + n_3 = n.}
\end{equation}

\subsection*{Summary for Part (b)}
\begin{itemize}
  \item \textbf{Energy:}\quad \(E_n = \hbar \omega (n + \tfrac{3}{2})\).
  \item \textbf{Degeneracy:}\quad \(g_n = \frac{(n+1)(n+2)}{2}\).
  \item \textbf{Raising-Operator Representation:}\quad states in the \(n\)-th manifold are generated by distributing \(n\) creation-operator actions among the three coordinates.
\end{itemize}





\section*{Question 2: Hermiticity Conditions in Spherical Coordinates (Expanded Explanation)}

We have the operator
\begin{equation}
\hat{A} = r \frac{\partial^2}{\partial r^2} + a \frac{\partial}{\partial r}
+ \frac{1}{r}\Bigl(\frac{\partial^2}{\partial \theta^2} + b \,\cot(\theta)\,\frac{\partial}{\partial \theta}\Bigr),
\end{equation}
where \(a,b \in \mathbb{C}\). We want to find for which values of \(a,b\) this operator is Hermitian in \(L^2(\mathbb{R}^3)\) with the measure \(r^2 \sin\theta\, dr\, d\theta\, d\phi\). The assumption is that wavefunctions vanish sufficiently fast as \(r\to 0, r\to\infty,\) and that everything is well-behaved at \(\theta=0,\theta=\pi\).

\subsection*{Why Hermiticity Requires Integration by Parts}
To say \(\hat{A}\) is Hermitian, we require
\begin{equation}
\int (\phi^* \hat{A}\,\psi)\, d^3x 
=
\int ((\hat{A}\,\phi)^*\,\psi)\, d^3x
\quad \text{for all }\phi,\psi\in L^2(\mathbb{R}^3),
\end{equation}
where \(d^3x = r^2\sin\theta\,dr\,d\theta\,d\phi\). If boundary terms from integration by parts fail to vanish or if the form of \(\hat{A}\) does not match its adjoint, Hermiticity fails.

\subsection*{Step 1: Radial Part in Detail}
Consider the radial piece:
\begin{equation}
\hat{A}_r = r\,\frac{\partial^2}{\partial r^2} \;+\; a\,\frac{\partial}{\partial r}.
\end{equation}

\paragraph{Integration Setup:}
We look at the radial integral (suppressing angular variables for the moment):
\[
\int_0^{\infty} dr\, r^2\,\phi^*(r)\,\Bigl(r\,\psi''(r) + a\,\psi'(r)\Bigr),
\]
where \(\psi'(r) = \tfrac{d\psi}{dr}\). We assume boundary conditions such that \(\phi,\psi\to 0\) as \(r\to0\) or \(\infty\), so that boundary terms vanish.

\paragraph{Integration by Parts:}
1. First, note that \(r\,\psi''(r)\) can produce terms of the form \(\frac{d}{dr}\bigl(r\,\psi'(r)\bigr)\) etc. Doing a full integration by parts carefully indicates that the operator becomes self-adjoint only if the coefficient in front of \(\tfrac{\partial}{\partial r}\) is real.
2. Similarly, the usual radial part of the Laplacian in spherical coordinates is known to be self-adjoint if it looks like \(\frac{1}{r^2}\frac{d}{dr}\bigl(r^2\frac{d}{dr}\bigr)\). Written out, that equates to certain terms that match \(\hat{A}_r\) only if \(a\) is real.

\textbf{Conclusion from radial part:} \(a \in \mathbb{R}\).

\subsection*{Step 2: Angular Part in Detail}
Now consider the angular part:
\begin{equation}
\frac{1}{r}\Bigl(\frac{\partial^2}{\partial \theta^2} + b\,\cot(\theta)\,\frac{\partial}{\partial \theta}\Bigr).
\end{equation}
Focusing on integration over \(\theta\in[0,\pi]\) with measure \(\sin\theta\,d\theta\) (and also integrating over \(\phi\), but that part is trivial if \(\hat{A}\) does not depend on \(\phi\)):

\[
\int_0^\pi d\theta\,\sin\theta\,\phi^*(\theta)\Bigl(\psi''_{\theta}(\theta) + b\,\cot(\theta)\,\psi'_{\theta}(\theta)\Bigr),
\]
where \(\psi'_{\theta}(\theta)=\tfrac{\partial\psi}{\partial \theta}\).
The standard angular part of the Laplacian is
\begin{equation}
\frac{1}{\sin\theta}\frac{\partial}{\partial \theta}\Bigl(\sin\theta\,\frac{\partial}{\partial \theta}\Bigr)
=
\frac{\partial^2}{\partial \theta^2}
+
\cot(\theta)\,\frac{\partial}{\partial \theta}.
\end{equation}
Clearly, that forces the coefficient of \(\cot(\theta)\,\tfrac{\partial}{\partial \theta}\) to be exactly \(1\). If \(b\neq 1\), we get a mismatch that leads to leftover factors or boundary terms, breaking Hermiticity. Also, for it to be truly self-adjoint, we want \(b\) real as well.

\textbf{Conclusion from angular part:} \(b = 1\) (and real).

\subsection*{Final Conclusion}
Thus, the operator \(\hat{A}\) will be Hermitian exactly if
\begin{equation}
\boxed{\,a \in \mathbb{R}, \quad b = 1.}
\end{equation}
That mirrors the known structure of the radial and angular parts of the Laplacian in spherical coordinates and ensures no boundary terms survive under integration by parts.




\end{document}
