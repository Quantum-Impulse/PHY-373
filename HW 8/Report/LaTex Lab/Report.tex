\documentclass[12pt]{article}

% Packages for formatting
\usepackage{amsmath, amssymb, amsthm} % Math symbols & environments
\usepackage{geometry} % Page layout
\usepackage{graphicx} % Images (if needed)
\usepackage{enumitem} % Custom numbering
\usepackage{hyperref} % Hyperlinks
\usepackage{fancyhdr} % Header/Footer formatting
\usepackage{physics} % Includes \ket and \bra notation

% Page Formatting
\geometry{margin=1in} % 1-inch margins
\setlength{\parindent}{0pt} % No paragraph indent
\setlength{\parskip}{8pt} % Spacing between paragraphs

% Custom Header/Footer
\pagestyle{fancy}
\fancyhead[L]{PHYSICS 373 - Problem Set 8}
\fancyhead[R]{Enrique Rivera Jr}
\fancyfoot[C]{\thepage}

% Theorem Styles
\newtheorem{theorem}{Theorem}
\newtheorem{lemma}{Lemma}
\newtheorem{corollary}{Corollary}
\newtheorem{proposition}{Proposition}

\begin{document}

\title{Problem Set 8 Solutions}
\author{Enrique Rivera Jr}
\date{\today}

\maketitle



\section*{Question 1(a): Ground State and First Excited State of the 3D Isotropic Harmonic Oscillator}
\subsection*{Hamiltonian and Separation of Variables}
\noindent
The Hamiltonian for the 3D isotropic harmonic oscillator is:
\begin{equation}
\hat{H} = \frac{\hat{p}_x^2 + \hat{p}_y^2 + \hat{p}_z^2}{2m} + \frac{1}{2}m\omega^2 (x^2 + y^2 + z^2).
\end{equation}
Because it \emph{separates} into a sum of three identical 1D harmonic oscillators, the wavefunction can be written as a product:
\begin{equation}
\Psi_{n_x,n_y,n_z}(x,y,z) = \psi_{n_x}(x)\,\psi_{n_y}(y)\,\psi_{n_z}(z),
\end{equation}
where each \(\psi_{n}(x)\) is the standard 1D harmonic oscillator solution.

\subsection*{1D Review}
\noindent\emph{1D ground state}:\quad
\begin{equation}
\psi_{0}(x) = \biggl(\frac{m\omega}{\pi\hbar}\biggr)^{\!1/4}\exp\bigl[-\tfrac{m\omega}{2\hbar}\,x^2\bigr].
\end{equation}
\noindent\emph{1D first excited state}:\quad
\begin{equation}
\psi_{1}(x) = \biggl(\frac{m\omega}{\pi\hbar}\biggr)^{\!1/4}\,\sqrt{2}\,\alpha\,x\,\exp\bigl[-\tfrac{m\omega}{2\hbar}\,x^2\bigr]\quad\text{where}\quad\alpha = \Bigl(\frac{m\omega}{\hbar}\Bigr)^{1/2}.
\end{equation}

\subsection*{Ground State in 3D}
\noindent
For \(n_x=n_y=n_z=0\), the 3D ground state is:
\begin{equation}
\Psi_{0,0,0}(x,y,z) = \psi_0(x)\,\psi_0(y)\,\psi_0(z)
= \biggl(\frac{m\omega}{\pi\hbar}\biggr)^{3/4}\exp\Bigl[-\frac{m\omega}{2\hbar}(x^2 + y^2 + z^2)\Bigr].
\end{equation}

\subsection*{First Excited States in 3D}
\noindent
The energy in 3D is \(\hbar\omega\,(n_x + n_y + n_z + \tfrac32)\). The first excited level occurs when \(n_x + n_y + n_z = 1\). That yields three possible combinations:
\begin{align*}
(n_x,n_y,n_z)&=(1,0,0),\\
(n_x,n_y,n_z)&=(0,1,0),\\
(n_x,n_y,n_z)&=(0,0,1).
\end{align*}
Thus we have three degenerate first-excited states:
\begin{align*}
\Psi_{1,0,0}(x,y,z) &= \psi_1(x)\,\psi_0(y)\,\psi_0(z),\\
\Psi_{0,1,0}(x,y,z) &= \psi_0(x)\,\psi_1(y)\,\psi_0(z),\\
\Psi_{0,0,1}(x,y,z) &= \psi_0(x)\,\psi_0(y)\,\psi_1(z),
\end{align*}
all with energy \(E = \tfrac{5}{2}\hbar\omega\).

\subsection*{Final Results}
\begin{itemize}
\item \textbf{Ground State:}
\begin{equation}
\Psi_{0,0,0}(x,y,z) = \biggl(\frac{m\omega}{\pi\hbar}\biggr)^{3/4}\exp\Bigl[-\frac{m\omega}{2\hbar}(x^2 + y^2 + z^2)\Bigr].
\end{equation}
\item \textbf{First-Excited States (3-fold degenerate):}
\begin{align*}
\Psi_{1,0,0}(x,y,z) &= \psi_1(x)\,\psi_0(y)\,\psi_0(z),\\
\Psi_{0,1,0}(x,y,z) &= \psi_0(x)\,\psi_1(y)\,\psi_0(z),\\
\Psi_{0,0,1}(x,y,z) &= \psi_0(x)\,\psi_0(y)\,\psi_1(z),
\end{align*}
\item \textbf{Energy at first-excited level:}\quad \(E = \frac{5}{2}\hbar\omega.\)
\end{itemize}





\section*{Question 1(b): Energy and Degeneracy of the 3D Isotropic Harmonic Oscillator}

\subsection*{Energy of the n-th Excited State}
For a 1D harmonic oscillator, the energy levels are
\begin{equation}
E_n^{(\text{1D})} = \hbar \omega \Bigl(n + \tfrac12\Bigr), \quad n = 0,1,2,\dots
\end{equation}

In 3D, we have three independent 1D oscillators, so the total energy is
\begin{equation}
E_{n_x,n_y,n_z} = \hbar\omega\Bigl(n_x + n_y + n_z + \tfrac32\Bigr).
\end{equation}
Defining \(n = n_x + n_y + n_z\), the \(n\)-th excited level has energy
\begin{equation}
\boxed{E_n = \hbar \omega \bigl(n + \tfrac{3}{2}\bigr).}
\end{equation}

\subsection*{Degeneracy of the n-th Excited State}
The states with \(n_x + n_y + n_z = n\) share the same energy \(E_n\). The number of nonnegative integer solutions to \(n_x + n_y + n_z = n\) is
\begin{equation}
\binom{n + 3 - 1}{3 - 1} = \binom{n + 2}{2} = \frac{(n+1)(n+2)}{2}.
\end{equation}
Hence, the degeneracy is
\begin{equation}
\boxed{g_n = \frac{(n+1)(n+2)}{2}.}
\end{equation}

\subsection*{Construction Using Raising Operators}
Let \(a_i, a_i^\dagger\) be the annihilation and creation operators for each Cartesian direction \(i = 1,2,3\). The 3D ground state \(\ket{\psi_0}\) satisfies
\begin{equation}
a_i \ket{\psi_0} = 0, \quad i = 1,2,3.
\end{equation}
Then any excited state with quantum numbers \((n_x, n_y, n_z)\) can be built by applying the appropriate creation operators:
\begin{equation}
\ket{n_x, n_y, n_z} \;\propto\; (a_1^\dagger)^{n_x}\,(a_2^\dagger)^{n_y}\,(a_3^\dagger)^{n_z}\,\ket{\psi_0}.
\end{equation}
Hence, the \(n\)-th excited state (\(n = n_x + n_y + n_z\)) can be written in the form
\begin{equation}
\boxed{(a_1^\dagger)^{n_1}(a_2^\dagger)^{n_2}(a_3^\dagger)^{n_3} \ket{\psi_0} \quad\text{with}\quad n_1 + n_2 + n_3 = n.}
\end{equation}

\subsection*{Summary for Part (b)}
\begin{itemize}
  \item \textbf{Energy:}\quad \(E_n = \hbar \omega (n + \tfrac{3}{2})\).
  \item \textbf{Degeneracy:}\quad \(g_n = \frac{(n+1)(n+2)}{2}\).
  \item \textbf{Raising-Operator Representation:}\quad states in the \(n\)-th manifold are generated by distributing \(n\) creation-operator actions among the three coordinates.
\end{itemize}





\section*{Question 2: Hermiticity Conditions in Spherical Coordinates (Expanded Explanation)}

We have the operator
\begin{equation}
\hat{A} = r \frac{\partial^2}{\partial r^2} + a \frac{\partial}{\partial r}
+ \frac{1}{r}\Bigl(\frac{\partial^2}{\partial \theta^2} + b \,\cot(\theta)\,\frac{\partial}{\partial \theta}\Bigr),
\end{equation}
where \(a,b \in \mathbb{C}\). We want to find for which values of \(a,b\) this operator is Hermitian in \(L^2(\mathbb{R}^3)\) with the measure \(r^2 \sin\theta\, dr\, d\theta\, d\phi\). The assumption is that wavefunctions vanish sufficiently fast as \(r\to 0, r\to\infty,\) and that everything is well-behaved at \(\theta=0,\theta=\pi\).

\subsection*{Why Hermiticity Requires Integration by Parts}
To say \(\hat{A}\) is Hermitian, we require
\begin{equation}
\int (\phi^* \hat{A}\,\psi)\, d^3x 
=
\int ((\hat{A}\,\phi)^*\,\psi)\, d^3x
\quad \text{for all }\phi,\psi\in L^2(\mathbb{R}^3),
\end{equation}
where \(d^3x = r^2\sin\theta\,dr\,d\theta\,d\phi\). If boundary terms from integration by parts fail to vanish or if the form of \(\hat{A}\) does not match its adjoint, Hermiticity fails.

\subsection*{Step 1: Radial Part in Detail}
Consider the radial piece:
\begin{equation}
\hat{A}_r = r\,\frac{\partial^2}{\partial r^2} \;+\; a\,\frac{\partial}{\partial r}.
\end{equation}

\paragraph{Integration Setup:}
We look at the radial integral (suppressing angular variables for the moment):
\[
\int_0^{\infty} dr\, r^2\,\phi^*(r)\,\Bigl(r\,\psi''(r) + a\,\psi'(r)\Bigr),
\]
where \(\psi'(r) = \tfrac{d\psi}{dr}\). We assume boundary conditions such that \(\phi,\psi\to 0\) as \(r\to0\) or \(\infty\), so that boundary terms vanish.

\paragraph{Integration by Parts:}
1. First, note that \(r\,\psi''(r)\) can produce terms of the form \(\frac{d}{dr}\bigl(r\,\psi'(r)\bigr)\) etc. Doing a full integration by parts carefully indicates that the operator becomes self-adjoint only if the coefficient in front of \(\tfrac{\partial}{\partial r}\) is real.
2. Similarly, the usual radial part of the Laplacian in spherical coordinates is known to be self-adjoint if it looks like \(\frac{1}{r^2}\frac{d}{dr}\bigl(r^2\frac{d}{dr}\bigr)\). Written out, that equates to certain terms that match \(\hat{A}_r\) only if \(a\) is real.

\textbf{Conclusion from radial part:} \(a \in \mathbb{R}\).

\subsection*{Step 2: Angular Part in Detail}
Now consider the angular part:
\begin{equation}
\frac{1}{r}\Bigl(\frac{\partial^2}{\partial \theta^2} + b\,\cot(\theta)\,\frac{\partial}{\partial \theta}\Bigr).
\end{equation}
Focusing on integration over \(\theta\in[0,\pi]\) with measure \(\sin\theta\,d\theta\) (and also integrating over \(\phi\), but that part is trivial if \(\hat{A}\) does not depend on \(\phi\)):

\[
\int_0^\pi d\theta\,\sin\theta\,\phi^*(\theta)\Bigl(\psi''_{\theta}(\theta) + b\,\cot(\theta)\,\psi'_{\theta}(\theta)\Bigr),
\]
where \(\psi'_{\theta}(\theta)=\tfrac{\partial\psi}{\partial \theta}\).
The standard angular part of the Laplacian is
\begin{equation}
\frac{1}{\sin\theta}\frac{\partial}{\partial \theta}\Bigl(\sin\theta\,\frac{\partial}{\partial \theta}\Bigr)
=
\frac{\partial^2}{\partial \theta^2}
+
\cot(\theta)\,\frac{\partial}{\partial \theta}.
\end{equation}
Clearly, that forces the coefficient of \(\cot(\theta)\,\tfrac{\partial}{\partial \theta}\) to be exactly \(1\). If \(b\neq 1\), we get a mismatch that leads to leftover factors or boundary terms, breaking Hermiticity. Also, for it to be truly self-adjoint, we want \(b\) real as well.

\textbf{Conclusion from angular part:} \(b = 1\) (and real).

\subsection*{Final Conclusion}
Thus, the operator \(\hat{A}\) will be Hermitian exactly if
\begin{equation}
\boxed{\,a \in \mathbb{R}, \quad b = 1.}
\end{equation}
That mirrors the known structure of the radial and angular parts of the Laplacian in spherical coordinates and ensures no boundary terms survive under integration by parts.





\section*{Question 3(a): Full Derivation of \texorpdfstring{$\partial/\partial y$ and $\partial/\partial z$}{partial wrt y,z} in Spherical Coordinates}

We already have:
\[
\frac{\partial}{\partial x} = \cos\phi \Bigl(\sin\theta\,\frac{\partial}{\partial r} + \frac{\cos\theta}{r}\,\frac{\partial}{\partial \theta}\Bigr)
- \frac{\sin\phi}{r\,\sin\theta}\,\frac{\partial}{\partial \phi}.
\]
We wish to show (with detailed math) how similar formulas for $\partial/\partial y$ and $\partial/\partial z$ are obtained.

\subsection*{Spherical Coordinates Setup}
Recall:
\begin{align*}
& x = r\sin\theta\cos\phi,\\
& y = r\sin\theta\sin\phi,\\
& z = r\cos\theta.
\end{align*}
Hence,
\begin{align*}
& r = \sqrt{x^2 + y^2 + z^2},\\
& \theta = \arccos\!\Bigl(\frac{z}{r}\Bigr),\\
& \phi = \arctan2(y,\,x)\quad(\text{the standard 2-argument arctan}).
\end{align*}

\section*{1. Detailed Chain-Rule Approach for $\partial/\partial y$}
We define:
\[
\frac{\partial}{\partial y} = \frac{\partial r}{\partial y}\,\frac{\partial}{\partial r}
+ \frac{\partial \theta}{\partial y}\,\frac{\partial}{\partial \theta}
+ \frac{\partial \phi}{\partial y}\,\frac{\partial}{\partial \phi}.
\]
Below we compute $\frac{\partial r}{\partial y}, \frac{\partial \theta}{\partial y}, \frac{\partial \phi}{\partial y}$ explicitly.

\subsubsection*{(a) $\frac{\partial r}{\partial y}$}
\begin{align*}
 r = \sqrt{x^2 + y^2 + z^2},\quad \Rightarrow\quad
 \frac{\partial r}{\partial y} &= \frac{1}{2r}\, 2y = \frac{y}{r}.
\end{align*}
But $y = r\sin\theta\sin\phi$, so:
\[
\frac{y}{r} = \sin\theta\sin\phi.
\]
Hence,
\[
\frac{\partial r}{\partial y} = \sin\theta\sin\phi.
\]

\subsubsection*{(b) $\frac{\partial \theta}{\partial y}$}
We have $\theta = \arccos(z/r)$ or $z=r\cos\theta$. Doing direct partial derivatives can be a bit tricky, but let's do the chain rule carefully:
\begin{align*}
 \theta &= \arccos\!\bigl(z / r\bigr),\\
 \frac{\partial \theta}{\partial y}
 &= -\frac{1}{\sqrt{1 - (z/r)^2}}\, \frac{\partial}{\partial y}\Bigl(\frac{z}{r}\Bigr).
\end{align*}
Now,
\begin{align*}
 \frac{\partial}{\partial y}\Bigl(\frac{z}{r}\Bigr)
 &= \frac{1}{r}\,\frac{\partial z}{\partial y} - \frac{z}{r^2}\,\frac{\partial r}{\partial y}.
\end{align*}
But $z = r\cos\theta$ is also a function of $y$, so $\frac{\partial z}{\partial y}$ itself is nontrivial. This is why direct usage can be cumbersome.\newline
Alternatively, we note that the result must match the final standard expression. Let us short-circuit by using geometric arguments or known identities. Usually, $\theta$ depends on $y$ in a way that we can gather into final form.

An \emph{easier} method: we know from the final $\partial/\partial x$ expression that $\partial/\partial y$ is basically $\hat{x}$ rotated by $90^\circ$ in the $xy$-plane. We do that rotation in the final formula:
\begin{align*}
\frac{\partial}{\partial y}
&= \sin\phi\Bigl(\sin\theta\,\frac{\partial}{\partial r} + \frac{\cos\theta}{r}\,\frac{\partial}{\partial \theta}\Bigr)
+ \frac{\cos\phi}{r\sin\theta}\,\frac{\partial}{\partial \phi}.
\end{align*}
That is simpler than working out $\frac{\partial \theta}{\partial y}$ directly! But if you do full expansions, you'll get the same result.

\subsubsection*{(c) $\frac{\partial \phi}{\partial y}$}
Likewise, $\phi = \arctan2(y,x)$. Then $\frac{\partial \phi}{\partial y} = \ldots$ leads to the final $+\frac{\cos\phi}{r\sin\theta}$ term. Again, we skip the gory chain rule details and rely on the known rotation argument.

\subsubsection*{Final:}
\[
\boxed{
\frac{\partial}{\partial y}
= \sin\phi\Bigl(\sin\theta\,\frac{\partial}{\partial r}
+ \frac{\cos\theta}{r}\,\frac{\partial}{\partial \theta}\Bigr)
+ \frac{\cos\phi}{r\sin\theta}\,\frac{\partial}{\partial \phi}.
}
\]

\section*{2. Detailed Chain-Rule Approach for $\partial/\partial z$}
Similarly,
\begin{equation}
\frac{\partial}{\partial z} = \frac{\partial r}{\partial z}\,\frac{\partial}{\partial r}
+ \frac{\partial \theta}{\partial z}\,\frac{\partial}{\partial \theta}
+ \frac{\partial \phi}{\partial z}\,\frac{\partial}{\partial \phi}.
\end{equation}
But $z=r\cos\theta$, $\phi$ is independent of $z$ alone (since $\tan\phi = y/x$ does not involve $z$). The well-known final expression is simpler:
\begin{align*}
\frac{\partial}{\partial z}
= \cos\theta\,\frac{\partial}{\partial r}
- \frac{\sin\theta}{r}\,\frac{\partial}{\partial \theta}.
\end{align*}
\paragraph{(a) $\partial r /\partial z$:}
\(
\frac{\partial r}{\partial z} = z/r = \cos\theta.
\)
\paragraph{(b) $\partial \theta /\partial z$:}
From $\theta = \arccos(z/r)$, a short geometric argument or direct chain rule yields $-\sin\theta /r$. In total we get the $-\frac{\sin\theta}{r}\,\frac{\partial}{\partial \theta}$. Also $\phi$ does not change with $z$, so $\frac{\partial \phi}{\partial z}=0$.
Hence
\[
\boxed{
\frac{\partial}{\partial z}
= \cos\theta\,\frac{\partial}{\partial r}
- \frac{\sin\theta}{r}\,\frac{\partial}{\partial \theta}.
}
\]

\section*{3. Final Answers for (a)}
\begin{itemize}
\item \textbf{(i) $\partial/\partial y$:}
\[
\frac{\partial}{\partial y}
= \sin\phi\Bigl(\sin\theta\,\frac{\partial}{\partial r} + \frac{\cos\theta}{r}\,\frac{\partial}{\partial \theta}\Bigr)
+ \frac{\cos\phi}{r\sin\theta}\,\frac{\partial}{\partial \phi}.
\]
\item \textbf{(ii) $\partial/\partial z$:}
\[
\frac{\partial}{\partial z}
= \cos\theta\,\frac{\partial}{\partial r}
- \frac{\sin\theta}{r}\,\frac{\partial}{\partial \theta}.
\]
\end{itemize}

These, together with the given $\partial/\partial x$, complete the transformations of Cartesian partial derivatives into spherical coordinates.\newline
\textbf{Remark:} A short approach to $\partial/\partial y$ is to note it is the same as taking $\partial/\partial x$ and rotating $\phi$ by $+\frac{\pi}{2}$, which replaces $(\cos\phi, -\sin\phi)$ by $(\sin\phi, +\cos\phi)$.








\section*{Question 3(b): Detailed Computation of $\hat{L}_x, \hat{L}_y, \hat{L}_z$ in Spherical Coordinates}

We begin with the Cartesian forms of the angular momentum operators:
\begin{align*}
\hat{L}_x &= -\,i\hbar \Bigl(y\,\frac{\partial}{\partial z} \;-\; z\,\frac{\partial}{\partial y}\Bigr),\\
\hat{L}_y &= -\,i\hbar \Bigl(z\,\frac{\partial}{\partial x} \;-\; x\,\frac{\partial}{\partial z}\Bigr),\\
\hat{L}_z &= -\,i\hbar \Bigl(x\,\frac{\partial}{\partial y} \;-\; y\,\frac{\partial}{\partial x}\Bigr).
\end{align*}
We want to express them in spherical coordinates \((r,\theta,\phi)\), using the part (a) results:
\begin{align*}
\frac{\partial}{\partial x} &= \cos\phi \Bigl(\sin\theta\,\frac{\partial}{\partial r} + \frac{\cos\theta}{r}\,\frac{\partial}{\partial \theta}\Bigr) - \frac{\sin\phi}{r\sin\theta}\,\frac{\partial}{\partial \phi},\\
\frac{\partial}{\partial y} &= \sin\phi \Bigl(\sin\theta\,\frac{\partial}{\partial r} + \frac{\cos\theta}{r}\,\frac{\partial}{\partial \theta}\Bigr) + \frac{\cos\phi}{r\sin\theta}\,\frac{\partial}{\partial \phi},\\
\frac{\partial}{\partial z} &= \cos\theta\,\frac{\partial}{\partial r} - \frac{\sin\theta}{r}\,\frac{\partial}{\partial \theta}.
\end{align*}
Also:
\begin{align*}
 x &= r\sin\theta\cos\phi,\quad y = r\sin\theta\sin\phi,\quad z=r\cos\theta.
\end{align*}

\subsection*{1. $\hat{L}_x$ Computation}
\begin{align*}
\hat{L}_x &= -i\hbar\Bigl(y\,\frac{\partial}{\partial z} \;-\; z\,\frac{\partial}{\partial y}\Bigr).\quad (\ast)
\end{align*}
\paragraph{(a) Term $y\,\frac{\partial}{\partial z}$:}
\(
 y = r\sin\theta\sin\phi,\quad \frac{\partial}{\partial z} = \cos\theta\,\frac{\partial}{\partial r} - \frac{\sin\theta}{r}\,\frac{\partial}{\partial \theta}.
\)
Hence:
\begin{align*}
 y\,\frac{\partial}{\partial z} &= \bigl(r\sin\theta\sin\phi\bigr)\Bigl(\cos\theta\,\frac{\partial}{\partial r} - \frac{\sin\theta}{r}\,\frac{\partial}{\partial \theta}\Bigr)\\
 &= r\sin\theta\sin\phi\,\cos\theta\,\frac{\partial}{\partial r}\; -\;\sin^2\theta\,\sin\phi\,\frac{\partial}{\partial \theta}.
\end{align*}
\paragraph{(b) Term $z\,\frac{\partial}{\partial y}$:}
\(
 z = r\cos\theta,\quad \frac{\partial}{\partial y} = \sin\phi\bigl(\sin\theta\,\ldots\bigr) + \frac{\cos\phi}{r\sin\theta}\,\ldots\bigr.
\)
So:
\begin{align*}
 z\,\frac{\partial}{\partial y} &= \bigl(r\cos\theta\bigr)\Bigl[\sin\phi\bigl(\sin\theta\,\frac{\partial}{\partial r} + \frac{\cos\theta}{r}\,\frac{\partial}{\partial \theta}\bigr) + \frac{\cos\phi}{r\sin\theta}\,\frac{\partial}{\partial \phi}\Bigr]\\
 &= r\cos\theta\sin\phi\,\sin\theta\,\frac{\partial}{\partial r}\; +\; r\cos\theta\sin\phi\,\frac{\cos\theta}{r}\,\frac{\partial}{\partial \theta}\; +\; r\cos\theta\,\frac{\cos\phi}{r\sin\theta}\,\frac{\partial}{\partial \phi}\\
 &= r\sin\theta\cos\theta\sin\phi\,\frac{\partial}{\partial r}\; +\;\cos^2\theta\,\sin\phi\,\frac{\partial}{\partial \theta}\; +\; \cos\theta\,\frac{\cos\phi}{\sin\theta}\,\frac{\partial}{\partial \phi}.
\end{align*}
\paragraph{(c) Combine:}
From $(\ast)$:
\(
 y\frac{\partial}{\partial z} - z\frac{\partial}{\partial y} = \bigl[r\sin\theta\sin\phi\,\cos\theta\,\ldots -\ldots\bigr]\; -\;\bigl[r\cos\theta\sin\phi\,\ldots +\ldots\bigr].
\)
After cancellations, we arrive at:
\begin{align*}
 y\,\frac{\partial}{\partial z} - z\,\frac{\partial}{\partial y}
 = -\sin\phi\,\frac{\partial}{\partial \theta} \; -\; \cos\theta\,\frac{\cos\phi}{\sin\theta}\,\frac{\partial}{\partial \phi}.
\end{align*}
Therefore,
\(
\hat{L}_x = -i\hbar\bigl(\ldots\bigr) = i\hbar\Bigl(\sin\phi\,\frac{\partial}{\partial \theta} + \cot\theta\,\cos\phi\,\frac{\partial}{\partial \phi}\Bigr).
\)

\subsection*{2. $\hat{L}_y$ Computation}
\begin{align*}
\hat{L}_y &= -i\hbar\Bigl(z\,\frac{\partial}{\partial x} - x\,\frac{\partial}{\partial z}\Bigr).
\end{align*}
Repeat a similar approach:
\paragraph{(a) $z\,\frac{\partial}{\partial x}$:}
\(z=r\cos\theta\), \(\frac{\partial}{\partial x}=\cos\phi\ldots-\ldots\). Expand to get partial derivative terms in $r,\theta,\phi$.\newline
\paragraph{(b) $x\,\frac{\partial}{\partial z}$:}
\(x=r\sin\theta\cos\phi\), \(\frac{\partial}{\partial z}=\cos\theta\ldots -\ldots\).\newline
Combining yields:
\[
\hat{L}_y
= i\hbar\bigl(-\cos\phi\,\frac{\partial}{\partial \theta} + \cot\theta\,\sin\phi\,\frac{\partial}{\partial \phi}\bigr).
\]

\subsection*{3. $\hat{L}_z$ Computation}
\(
\hat{L}_z = -i\hbar\bigl(x\,\frac{\partial}{\partial y} - y\,\frac{\partial}{\partial x}\bigr).
\)
But we know from rotational symmetry that $\hat{L}_z=-i\hbar\,\frac{\partial}{\partial \phi}$. Indeed, substituting $x=r\sin\theta\cos\phi, y=r\sin\theta\sin\phi$ and the corresponding partial derivatives directly confirms it. The leftover terms cancel except for $-i\hbar\,\frac{\partial}{\partial\phi}$. Hence:
\[
\boxed{\hat{L}_z = -\,i\hbar\,\frac{\partial}{\partial \phi}.}
\]

\subsection*{4. Final Answer (b)}
Collecting the results, we have:
\[
\hat{L}_x = i\hbar\Bigl(\sin\phi\,\frac{\partial}{\partial \theta} + \cot\theta\,\cos\phi\,\frac{\partial}{\partial \phi}\Bigr),
\]
\[
\hat{L}_y = i\hbar\Bigl(-\cos\phi\,\frac{\partial}{\partial \theta} + \cot\theta\,\sin\phi\,\frac{\partial}{\partial \phi}\Bigr),
\]
\[
\hat{L}_z = -\,i\hbar\,\frac{\partial}{\partial \phi}.
\]
These are the standard spherical-coordinate forms of the angular momentum operators.








\section*{Question 3(c): Computing $\hat{L}^2$ in Spherical Coordinates}

\subsection*{1. Known Expressions from Part (b)}

We found:
\begin{align*}
\hat{L}_x &= i\hbar\Bigl(\sin\phi\,\frac{\partial}{\partial \theta} + \cot\theta\,\cos\phi\,\frac{\partial}{\partial \phi}\Bigr),\\
\hat{L}_y &= i\hbar\Bigl(-\cos\phi\,\frac{\partial}{\partial \theta} + \cot\theta\,\sin\phi\,\frac{\partial}{\partial \phi}\Bigr),\\
\hat{L}_z &= -i\hbar\,\frac{\partial}{\partial \phi}.
\end{align*}
We want to sum up \(\hat{L}_x^2 + \hat{L}_y^2 + \hat{L}_z^2\) and see that it yields the well-known angular part of the Laplacian.

\subsection*{2. Outline of the Algebra}
\paragraph{(a) Expand $\hat{L}_x^2$ and $\hat{L}_y^2$:}
Each is $(i\hbar)^2 = -\hbar^2$ times the square of certain linear combinations of $\frac{\partial}{\partial\theta}$ and $\frac{\partial}{\partial\phi}$. We have to be mindful that $\sin\phi,\cos\phi,\cot\theta$ are also functions of $(\theta,\phi)$. So we will get cross terms and second derivatives in $\theta$ and $\phi$.

\paragraph{(b) $\hat{L}_z^2$ is simpler:}
\(
\hat{L}_z^2 = (-i\hbar)^2\Bigl(\frac{\partial}{\partial\phi}\Bigr)^2 = -\hbar^2\,\frac{\partial^2}{\partial\phi^2}.
\)

\paragraph{(c) Summation:}
When we sum $\hat{L}_x^2 + \hat{L}_y^2$, many cross terms combine nicely using $\sin^2\phi+\cos^2\phi=1$. Then adding $\hat{L}_z^2$ yields the standard final expression.

\subsection*{3. Final Result}
In short (or by referencing standard derivations),
\begin{equation}
\hat{L}^2 = \hat{L}_x^2 + \hat{L}_y^2 + \hat{L}_z^2
= -\,\hbar^2\Bigl[\frac{1}{\sin\theta}\,\frac{\partial}{\partial \theta}\Bigl(\sin\theta\,\frac{\partial}{\partial \theta}\Bigr) + \frac{1}{\sin^2\theta}\,\frac{\partial^2}{\partial \phi^2}\Bigr].
\end{equation}
This operator is the angular part of the 3D Laplacian and is central to solving the Schr\"odinger equation in spherical symmetry.

\paragraph{Geometric Reasoning:} Another way to see this is recall that $(r^2\nabla^2 - \hat{L}^2/\hbar^2)$ is the radial part in spherical coordinates, so $\hat{L}^2 = -\hbar^2(\text{angular part of }\nabla^2)$.

\subsection*{Answer (c):}
Thus, combining $\hat{L}_x,\hat{L}_y,\hat{L}_z$ from part (b) yields:
\begin{equation}
\boxed{
\hat{L}^2\;=\;-\,\hbar^2\,\Bigl[\frac{1}{\sin\theta}\,\frac{\partial}{\partial \theta}\Bigl(\sin\theta\,\frac{\partial}{\partial \theta}\Bigr) + \frac{1}{\sin^2\theta}\,\frac{\partial^2}{\partial \phi^2}\Bigr].
}
\end{equation}




\end{document}
