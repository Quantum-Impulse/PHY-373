\documentclass[12pt]{article}

% Packages for formatting
\usepackage{amsmath, amssymb, amsthm} % Math symbols & environments
\usepackage{geometry} % Page layout
\usepackage{graphicx} % Images (if needed)
\usepackage{enumitem} % Custom numbering
\usepackage{hyperref} % Hyperlinks
\usepackage{fancyhdr} % Header/Footer formatting
\usepackage{physics} % Includes \ket and \bra notation

% Page Formatting
\geometry{margin=1in} % 1-inch margins
\setlength{\parindent}{0pt} % No paragraph indent
\setlength{\parskip}{8pt} % Spacing between paragraphs

% Custom Header/Footer
\pagestyle{fancy}
\fancyhead[L]{PHYSICS 373 - Problem Set 8}
\fancyhead[R]{Enrique Rivera Jr}
\fancyfoot[C]{\thepage}

% Theorem Styles
\newtheorem{theorem}{Theorem}
\newtheorem{lemma}{Lemma}
\newtheorem{corollary}{Corollary}
\newtheorem{proposition}{Proposition}

\begin{document}

\title{Problem Set 8 Solutions}
\author{Enrique Rivera Jr}
\date{\today}

\maketitle





\section*{Question 1(a): Ground State and First Excited State of the 3D Isotropic Harmonic Oscillator}
\subsection*{Hamiltonian and Separation of Variables}
\noindent
The Hamiltonian for the 3D isotropic harmonic oscillator is:
\begin{equation}
\hat{H} = \frac{\hat{p}_x^2 + \hat{p}_y^2 + \hat{p}_z^2}{2m} + \frac{1}{2}m\omega^2 (x^2 + y^2 + z^2).
\end{equation}
Because it \emph{separates} into a sum of three identical 1D harmonic oscillators, the wavefunction can be written as a product:
\begin{equation}
\Psi_{n_x,n_y,n_z}(x,y,z) = \psi_{n_x}(x)\,\psi_{n_y}(y)\,\psi_{n_z}(z),
\end{equation}
where each \(\psi_{n}(x)\) is the standard 1D harmonic oscillator solution.

\subsection*{1D Review}
\noindent\emph{1D ground state}:\quad
\begin{equation}
\psi_{0}(x) = \biggl(\frac{m\omega}{\pi\hbar}\biggr)^{\!1/4}\exp\bigl[-\tfrac{m\omega}{2\hbar}\,x^2\bigr].
\end{equation}
\noindent\emph{1D first excited state}:\quad
\begin{equation}
\psi_{1}(x) = \biggl(\frac{m\omega}{\pi\hbar}\biggr)^{\!1/4}\,\sqrt{2}\,\alpha\,x\,\exp\bigl[-\tfrac{m\omega}{2\hbar}\,x^2\bigr]\quad\text{where}\quad\alpha = \Bigl(\frac{m\omega}{\hbar}\Bigr)^{1/2}.
\end{equation}

\subsection*{Ground State in 3D}
\noindent
For \(n_x=n_y=n_z=0\), the 3D ground state is:
\begin{equation}
\Psi_{0,0,0}(x,y,z) = \psi_0(x)\,\psi_0(y)\,\psi_0(z)
= \biggl(\frac{m\omega}{\pi\hbar}\biggr)^{3/4}\exp\Bigl[-\frac{m\omega}{2\hbar}(x^2 + y^2 + z^2)\Bigr].
\end{equation}

\subsection*{First Excited States in 3D}
\noindent
The energy in 3D is \(\hbar\omega\,(n_x + n_y + n_z + \tfrac32)\). The first excited level occurs when \(n_x + n_y + n_z = 1\). That yields three possible combinations:
\begin{align*}
(n_x,n_y,n_z)&=(1,0,0),\\
(n_x,n_y,n_z)&=(0,1,0),\\
(n_x,n_y,n_z)&=(0,0,1).
\end{align*}
Thus we have three degenerate first-excited states:
\begin{align*}
\Psi_{1,0,0}(x,y,z) &= \psi_1(x)\,\psi_0(y)\,\psi_0(z),\\
\Psi_{0,1,0}(x,y,z) &= \psi_0(x)\,\psi_1(y)\,\psi_0(z),\\
\Psi_{0,0,1}(x,y,z) &= \psi_0(x)\,\psi_0(y)\,\psi_1(z),
\end{align*}
all with energy \(E = \tfrac{5}{2}\hbar\omega\).

\subsection*{Final Results}
\begin{itemize}
\item \textbf{Ground State:}
\begin{equation}
\Psi_{0,0,0}(x,y,z) = \biggl(\frac{m\omega}{\pi\hbar}\biggr)^{3/4}\exp\Bigl[-\frac{m\omega}{2\hbar}(x^2 + y^2 + z^2)\Bigr].
\end{equation}
\item \textbf{First-Excited States (3-fold degenerate):}
\begin{align*}
\Psi_{1,0,0}(x,y,z) &= \psi_1(x)\,\psi_0(y)\,\psi_0(z),\\
\Psi_{0,1,0}(x,y,z) &= \psi_0(x)\,\psi_1(y)\,\psi_0(z),\\
\Psi_{0,0,1}(x,y,z) &= \psi_0(x)\,\psi_0(y)\,\psi_1(z),
\end{align*}
\item \textbf{Energy at first-excited level:}\quad \(E = \frac{5}{2}\hbar\omega.\)
\end{itemize}






\end{document}
