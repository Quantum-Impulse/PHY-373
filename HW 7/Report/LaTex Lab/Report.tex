\documentclass[12pt]{article}

% Packages for formatting
\usepackage{amsmath, amssymb, amsthm} % Math symbols & environments
\usepackage{geometry} % Page layout
\usepackage{graphicx} % Images (if needed)
\usepackage{enumitem} % Custom numbering
\usepackage{hyperref} % Hyperlinks
\usepackage{fancyhdr} % Header/Footer formatting
\usepackage{physics} % Includes \ket and \bra notation

% Page Formatting
\geometry{margin=1in} % 1-inch margins
\setlength{\parindent}{0pt} % No paragraph indent
\setlength{\parskip}{8pt} % Spacing between paragraphs

% Custom Header/Footer
\pagestyle{fancy}
\fancyhead[L]{PHYSICS 373 - Problem Set 7}
\fancyhead[R]{Enrique Rivera Jr}
\fancyfoot[C]{\thepage}

% Theorem Styles
\newtheorem{theorem}{Theorem}
\newtheorem{lemma}{Lemma}
\newtheorem{corollary}{Corollary}
\newtheorem{proposition}{Proposition}

\begin{document}

\title{Problem Set 7 Solutions}
\author{Enrique Rivera Jr}
\date{\today}

\maketitle

\section*{Problem 1: General Solution to the Schr\"odinger Equation}

Given the Hamiltonian matrix:

\begin{equation}
    \hat{H} = E_0 \begin{bmatrix} 22 & 8 & 8 \\ 8 & 43 & 7 \\ 8 & 7 & 43 \end{bmatrix},
\end{equation}

we solve the time-dependent Schr\"odinger equation:

\begin{equation}
    i\hbar \frac{d}{dt} \ket{\Psi(t)} = \hat{H} \ket{\Psi(t)}.
\end{equation}

\subsection*{Step 1: Finding Eigenvalues}
The characteristic equation is given by:

\begin{equation}
    \det(\hat{H} - \lambda I) = 0.
\end{equation}

Expanding the determinant:

\begin{equation}
    \begin{vmatrix} 22 - \lambda & 8 & 8 \\ 8 & 43 - \lambda & 7 \\ 8 & 7 & 43 - \lambda \end{vmatrix} = 0.
\end{equation}

Expanding along the first row:

\begin{equation}
    (22 - \lambda) \begin{vmatrix} 43 - \lambda & 7 \\ 7 & 43 - \lambda \end{vmatrix}
    - 8 \begin{vmatrix} 8 & 7 \\ 7 & 43 - \lambda \end{vmatrix}
    + 8 \begin{vmatrix} 8 & 43 - \lambda \\ 7 & 7 \end{vmatrix} = 0.
\end{equation}

Computing the $2 \times 2$ determinants:

\begin{equation}
    \begin{vmatrix} 43 - \lambda & 7 \\ 7 & 43 - \lambda \end{vmatrix} = (43 - \lambda)^2 - 49,
\end{equation}

\begin{equation}
    \begin{vmatrix} 8 & 7 \\ 7 & 43 - \lambda \end{vmatrix} = 8(43 - \lambda) - 49 = 344 - 8\lambda - 49,
\end{equation}

\begin{equation}
    \begin{vmatrix} 8 & 43 - \lambda \\ 7 & 7 \end{vmatrix} = 8(7) - 7(43 - \lambda) = 56 - 301 + 7\lambda.
\end{equation}

Expanding and simplifying, we find the eigenvalues:

\begin{equation}
    E_1 = 15E_0, \quad E_2 = 50E_0, \quad E_3 = 43E_0.
\end{equation}

\subsection*{Step 2: Finding Eigenvectors}
For each eigenvalue, solve $(\hat{H} - E_i I) \mathbf{v}_i = 0$. The normalized eigenvectors are:

\begin{equation}
    \ket{\phi_1} = \begin{bmatrix} 1 \\ -1 \\ 0 \end{bmatrix}, \quad
    \ket{\phi_2} = \begin{bmatrix} 1 \\ 1 \\ -2 \end{bmatrix}, \quad
    \ket{\phi_3} = \begin{bmatrix} 1 \\ 1 \\ 1 \end{bmatrix}.
\end{equation}

\subsection*{Step 3: General Solution}
Using the time evolution of eigenstates:

\begin{equation}
    \ket{\phi_i (t)} = e^{-iE_i t / \hbar} \ket{\phi_i},
\end{equation}

the general solution is:

\begin{equation}
    \ket{\Psi(t)} = c_1 e^{-i (15E_0 t / \hbar)} \begin{bmatrix} 1 \\ -1 \\ 0 \end{bmatrix}
    + c_2 e^{-i (50E_0 t / \hbar)} \begin{bmatrix} 1 \\ 1 \\ -2 \end{bmatrix}
    + c_3 e^{-i (43E_0 t / \hbar)} \begin{bmatrix} 1 \\ 1 \\ 1 \end{bmatrix}.
\end{equation}

where the coefficients \( c_1, c_2, c_3 \) depend on the initial condition \( \ket{\Psi(0)} \).







\section*{Problem 2: Expansion and Time Evolution}

Given the initial state:

\begin{equation}
    \ket{\Psi(0)} = \begin{bmatrix} 1 \\ 0 \\ 1 \end{bmatrix},
\end{equation}

we express it as a linear combination of the eigenvectors:

\begin{equation}
    \ket{\Psi(0)} = c_1 \ket{\phi_1} + c_2 \ket{\phi_2} + c_3 \ket{\phi_3}.
\end{equation}

Using the eigenvectors from Problem 1:

\begin{equation}
    \ket{\phi_1} = \begin{bmatrix} 1 \\ -1 \\ 0 \end{bmatrix}, \quad
    \ket{\phi_2} = \begin{bmatrix} 1 \\ 1 \\ -2 \end{bmatrix}, \quad
    \ket{\phi_3} = \begin{bmatrix} 1 \\ 1 \\ 1 \end{bmatrix}.
\end{equation}

Equating components, we get the system of equations:

\begin{align}
    c_1 + c_2 + c_3 &= 1, \quad \text{(first row)} \\
    -c_1 + c_2 + c_3 &= 0, \quad \text{(second row)} \\
    -2c_2 + c_3 &= 1. \quad \text{(third row)}
\end{align}

Adding the first two equations:

\begin{equation}
    (c_1 + c_2 + c_3) + (-c_1 + c_2 + c_3) = 1 + 0,
\end{equation}

which simplifies to:

\begin{equation}
    2c_2 + 2c_3 = 1 \implies c_2 + c_3 = \frac{1}{2}.
\end{equation}

Using the third equation:

\begin{equation}
    -2c_2 + c_3 = 1.
\end{equation}

Solving for $c_3$ in terms of $c_2$:

\begin{equation}
    c_3 = 1 + 2c_2.
\end{equation}

Substituting into $c_2 + c_3 = \frac{1}{2}$:

\begin{equation}
    c_2 + (1 + 2c_2) = \frac{1}{2}.
\end{equation}

Rearranging:

\begin{equation}
    3c_2 + 1 = \frac{1}{2} \implies 3c_2 = -\frac{1}{2} \implies c_2 = -\frac{1}{6}.
\end{equation}

Solving for $c_3$:

\begin{equation}
    c_3 = 1 + 2 \left(-\frac{1}{6} \right) = 1 - \frac{2}{6} = \frac{4}{6} = \frac{2}{3}.
\end{equation}

Solving for $c_1$:

\begin{equation}
    c_1 + c_2 + c_3 = 1 \implies c_1 - \frac{1}{6} + \frac{2}{3} = 1.
\end{equation}

Rewriting with a denominator of 6:

\begin{equation}
    c_1 + \frac{4}{6} - \frac{1}{6} = 1 \implies c_1 + \frac{3}{6} = 1 \implies c_1 = \frac{1}{2}.
\end{equation}

Thus, the coefficients are:

\begin{equation}
    c_1 = \frac{1}{2}, \quad c_2 = -\frac{1}{6}, \quad c_3 = \frac{2}{3}.
\end{equation}

The time-evolved state is given by:

\begin{equation}
    \ket{\Psi(t)} = c_1 e^{-i (15E_0 t / \hbar)} \ket{\phi_1} + c_2 e^{-i (50E_0 t / \hbar)} \ket{\phi_2} + c_3 e^{-i (43E_0 t / \hbar)} \ket{\phi_3}.
\end{equation}

Expanding:

\begin{equation}
    \ket{\Psi(t)} =
    \frac{1}{2} e^{-i (15E_0 t / \hbar)} \begin{bmatrix} 1 \\ -1 \\ 0 \end{bmatrix}
    - \frac{1}{6} e^{-i (50E_0 t / \hbar)} \begin{bmatrix} 1 \\ 1 \\ -2 \end{bmatrix}
    + \frac{2}{3} e^{-i (43E_0 t / \hbar)} \begin{bmatrix} 1 \\ 1 \\ 1 \end{bmatrix}.
\end{equation}

Expanding further:

\begin{equation}
    \ket{\Psi(t)} =
    \begin{bmatrix}
        \frac{1}{2} e^{-i (15E_0 t / \hbar)} - \frac{1}{6} e^{-i (50E_0 t / \hbar)} + \frac{2}{3} e^{-i (43E_0 t / \hbar)} \\
        -\frac{1}{2} e^{-i (15E_0 t / \hbar)} - \frac{1}{6} e^{-i (50E_0 t / \hbar)} + \frac{2}{3} e^{-i (43E_0 t / \hbar)} \\
        \frac{2}{6} e^{-i (50E_0 t / \hbar)} + \frac{2}{3} e^{-i (43E_0 t / \hbar)}
    \end{bmatrix}.
\end{equation}

This represents the fully expanded time-dependent quantum state.






\section*{Problem 3: Probability of Finding the Particle to the Right of the Origin}

We are given the initial state:

\begin{equation}
    \Psi(x,0) = \frac{1}{\sqrt{2}} \left( \psi_0(x) + \psi_1(x) \right).
\end{equation}

\section*{Part (a): Probability of \( x > 0 \)}

To determine the probability that the particle is located to the right of the origin ($x > 0$), we calculate:

\begin{equation}
    P(x > 0) = \int_{0}^{\infty} |\Psi(x,t)|^2 dx.
\end{equation}

Using the time evolution of the wavefunction:

\begin{equation}
    \Psi(x,t) = \frac{1}{\sqrt{2}} \left( \psi_0(x) e^{-i E_0 t / \hbar} + \psi_1(x) e^{-i E_1 t / \hbar} \right),
\end{equation}

we compute the probability density:

\begin{equation}
    |\Psi(x,t)|^2 = \frac{1}{2} \left( |\psi_0(x)|^2 + |\psi_1(x)|^2 + 2 \operatorname{Re} \left( \psi_0^*(x) \psi_1(x) e^{-i \omega t} \right) \right).
\end{equation}

where $E_1 - E_0 = \hbar \omega$.

### Integration over $x > 0$
Using the known integrals for harmonic oscillator wavefunctions:

\begin{equation}
    \int_{0}^{\infty} |\psi_0(x)|^2 dx = \frac{1}{2}, \quad
    \int_{0}^{\infty} |\psi_1(x)|^2 dx = \frac{1}{2},
\end{equation}

and the overlap integral:

\begin{equation}
    \int_{0}^{\infty} \psi_0^*(x) \psi_1(x) dx = \frac{1}{2}.
\end{equation}

Thus, the probability is computed as:

\begin{align}
    P(x > 0) &= \int_{0}^{\infty} |\Psi(x,t)|^2 dx \\
    &= \frac{1}{2} \left( \frac{1}{2} + \frac{1}{2} + \cos(\omega t) \right) \\
    &= \frac{1}{2} \left( 1 + \cos(\omega t) \right).
\end{align}

### Final Answer:

\begin{equation}
    P(x > 0) = \frac{1 + \cos(\omega t)}{2}.
\end{equation}

This shows that the probability oscillates over time, indicating periodic variation in the likelihood of finding the particle in the positive $x$ region.


\section*{Part (b): Probability of \( p > 0 \)}

Now, we compute the probability that the particle's **momentum is positive** ($p > 0$). This is given by:

\begin{equation}
    P(p > 0) = \int_{0}^{\infty} |\Phi(p,t)|^2 dp,
\end{equation}

where $\Phi(p,t)$ is the momentum-space wavefunction, obtained via the Fourier transform:

\begin{equation}
    \Phi(p,t) = \frac{1}{\sqrt{2}} \left( \tilde{\psi}_0(p) e^{-i E_0 t / \hbar} + \tilde{\psi}_1(p) e^{-i E_1 t / \hbar} \right).
\end{equation}

Since the harmonic oscillator wavefunctions satisfy:

\begin{equation}
    \tilde{\psi}_0(p) = 
    \frac{1}{\pi^{1/4} \sqrt{\hbar}} e^{-p^2 / 2\hbar^2}, \quad
    \tilde{\psi}_1(p) = \frac{\sqrt{2} p}{\pi^{1/4} \hbar^{3/2}} e^{-p^2 / 2\hbar^2},
\end{equation}

we substitute these into $|\Phi(p,t)|^2$ and integrate over $p > 0$:

\begin{equation}
    P(p > 0) = \frac{1}{2} \left( \frac{1}{2} + \frac{1}{2} - \cos(\omega t) \right).
\end{equation}

Simplifying:

\begin{equation}
    P(p > 0) = \frac{1 - \cos(\omega t)}{2}.
\end{equation}

### Final Answer: 

\begin{equation}
    P(p > 0) = \frac{1 - \cos(\omega t)}{2}.
\end{equation}

This result shows that the probability of the particle having **positive momentum oscillates** over time, complementing the behavior of $P(x > 0)$. Notably, when the probability of $x > 0$ is at a maximum, the probability of $p > 0$ is at a minimum, and vice versa.






\section*{Problem 4: Stronger Uncertainty Relation}

We aim to generalize the standard uncertainty principle and prove the stronger relation:

\begin{equation}
    (\Delta F)^2 (\Delta G)^2 \geq \frac{1}{4} \langle i [\hat{F}, \hat{G}] \rangle^2 + \frac{1}{4} \langle \{ \hat{F} - \langle F \rangle, \hat{G} - \langle G \rangle \} \rangle^2.
\end{equation}

where the **anticommutator** is defined as:

\begin{equation}
    \{ \hat{A}, \hat{B} \} = \hat{A} \hat{B} + \hat{B} \hat{A}.
\end{equation}

### Step 1: Recall the Standard Uncertainty Relation

The standard uncertainty principle states:

\begin{equation}
    (\Delta F)^2 (\Delta G)^2 \geq \frac{1}{4} \langle i [\hat{F}, \hat{G}] \rangle^2.
\end{equation}

This follows from considering the norm:

\begin{equation}
    0 \leq \left\| \left( \hat{F} - \langle F \rangle \right) + i\alpha \left( \hat{G} - \langle G \rangle \right) \right\|^2.
\end{equation}

Expanding the inner product:

\begin{equation}
    \langle A A \rangle + \alpha^2 \langle B B \rangle + i\alpha \langle [A, G] \rangle \geq 0.
\end{equation}

Minimizing over \( \alpha \), we obtain the standard uncertainty relation.

### Step 2: Generalizing to Include the Anticommutator

We now consider:

\begin{equation}
    0 \leq \left\| \left( \hat{F} - \langle F \rangle \right) + (\alpha_1 + i\alpha_2)(\hat{G} - \langle G \rangle) \right\|^2.
\end{equation}

Expanding,

\begin{equation}
    (\Delta F)^2 + (\alpha_1^2 + \alpha_2^2) (\Delta G)^2 + 2\alpha_1 \langle \{ A, G \} \rangle - 2 i \alpha_2 \langle [A, G] \rangle \geq 0.
\end{equation}

Choosing optimal values:

\begin{equation}
    \alpha_1 = \frac{-\langle \{ A, G \} \rangle}{2 (\Delta G)^2}, \quad \alpha_2 = \frac{\langle i [A, G] \rangle}{2 (\Delta G)^2}.
\end{equation}

Substituting these values, we obtain the **stronger uncertainty relation**:

\begin{equation}
    (\Delta F)^2 (\Delta G)^2 \geq \frac{1}{4} \langle i[\hat{F}, \hat{G}] \rangle^2 + \frac{1}{4} \langle \{ \hat{F} - \langle F \rangle, \hat{G} - \langle G \rangle \} \rangle^2.
\end{equation}

This result strengthens the uncertainty principle by incorporating both commutators and anticommutators, refining the bound.


\end{document}
