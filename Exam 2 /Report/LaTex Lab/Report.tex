\documentclass[12pt]{article}

% ----------------------------------------------------------------------
% basic packages
\usepackage{amsmath, amssymb, physics}
\usepackage[margin=1in]{geometry}
\usepackage{enumitem}

% ----------------------------------------------------------------------
% paragraph spacing; no indent
\setlength{\parindent}{0pt}
\setlength{\parskip}{6pt}

% handy vertical‑space macros
\newcommand{\qs}{\bigskip\bigskip}
\newcommand{\vv}{\medskip}

% -------------------------------------------------------  document  ---
\begin{document}

% =========================================================================
\begin{center}
  {\Large\bfseries PHYSICS 373 \\
               Second Mid‑Term Examination}\\[2ex]
  Time limit: \textbf{1 hour} \quad (calculator permitted)
\end{center}

% --------------------------------------------------------------------- %
\newpage
%%%%%%%%%%%%%%%%%%%%%%%%%%%%%%%%%%%%%%%%%%%%%%%%%%%%%%%%%%%%
%  Q 1 – True / False (with short explanations)           %
%%%%%%%%%%%%%%%%%%%%%%%%%%%%%%%%%%%%%%%%%%%%%%%%%%%%%%%%%%%%
\newpage
\section*{Question 1 — True / False (circle one)}

\begin{enumerate}[label=\textbf{\alph*)},itemsep=1.2\baselineskip]

%----------------------------------------------------------
\item \textbf{For the 3‑D isotropic harmonic oscillator, the $n=2$ energy level is 6‑fold
      degenerate.}

      \textbf{Answer:} \textbf{T}\\
      \textbf{Why?}  In three dimensions the degeneracy is
      $\displaystyle g_n=\binom{n+2}{2}$.  
      For $n=2$ one obtains $g_2=\binom{4}{2}=6$.\par
      \emph{Concept cue:}  The integer $n=n_1+n_2+n_3$ can be “distributed’’ among
      the three Cartesian directions in $\binom{n+2}{2}$ ways (stars‑and‑bars
      combinatorics).

%----------------------------------------------------------
\item \textbf{The spherical harmonic $Y_{\ell,\ell}(\theta,\phi)$ is independent of $\theta$.}

      \textbf{Answer:} \textbf{F}\\
      \textbf{Why?}  $Y_{\ell,\ell}\propto\sin^\ell\!\theta\,e^{i\ell\phi}$;
      the $\sin^\ell\!\theta$ factor clearly depends on~$\theta$.\par
      \emph{Concept cue:}  Highest‑weight harmonics peak on the equator
      ($\theta=\pi/2$) and vanish at the poles, so some $\theta$‑dependence is
      unavoidable.

%----------------------------------------------------------
\item \textbf{Adding the operator $\dfrac{eB}{2m_e}\,\hat L_z$ to the hydrogen Hamiltonian
      removes \emph{all} degeneracy of the $n=2$ levels.}

      \textbf{Answer:} \textbf{F}\\
      \textbf{Why?}  The Zeeman term lifts only the $m$ (magnetic) degeneracy; the
      $\ell$‑degeneracy between $2s$ and $2p$ states
      ($\ell=0$ vs.\ $\ell=1$) survives.\par
      \emph{Concept cue:}  A perturbation $\propto\hat L_z$ shifts states with
      different $m$’s but leaves states that share the same $m$ unaffected,
      regardless of~$\ell$.

%----------------------------------------------------------
\item \textbf{For any Hermitian operator $\hat F$ and any quantum state
      $\ket{\psi}$, the product $\Delta F\,\Delta L_z$ can reach zero.}

      \textbf{Answer:} \textbf{F}\\
      \textbf{Why?}  Both uncertainties vanish only if $\ket{\psi}$ is a simultaneous
      eigenstate of \(\hat F\) and \(\hat L_z\), which requires
      $[\hat F,\hat L_z]=0$.  For a generic Hermitian $\hat F$ this commutator is
      non‑zero, so the product cannot be zero (Heisenberg
      inequality).\par
      \emph{Concept cue:}  Zero uncertainties in two observables are possible
      \emph{only} when the observables commute and the state lies in their common
      eigenbasis.

%----------------------------------------------------------
\item \textbf{The spin‑$\tfrac{3}{2}$ matrices can be chosen purely real in the
      $S_z$ eigen‑basis.}

      \textbf{Answer:} \textbf{F}\\
      \textbf{Why?}  While $\hat S_z$ is real and diagonal in that basis,
      at least one of $\hat S_x,\hat S_y$ must contain imaginary entries;
      in particular $\hat S_y$ is proportional to the anti‑Hermitian part
      $(\hat S_+-\hat S_-)$ and cannot be made real simultaneously with
      $\hat S_x$.\par
      \emph{Concept cue:}  The spin commutation relation
      $[\hat S_x,\hat S_y]=i\hbar\hat S_z$ forces an $i$ in \emph{some}
      matrix entries in any basis.

\end{enumerate}

% --------------------------------------------------------------------- %
\newpage
{\large\bfseries 2.\; Ladder‑operator algebra and degeneracy}\qs

For the 3‑D isotropic harmonic oscillator we label Cartesian
states by $\ket{n_1,n_2,n_3}$ with
$n=n_1+n_2+n_3$ and spherical states by $\ket{n; \ell, m}$.

\begin{enumerate}[label=\textbf{\alph*)}, leftmargin=1.2cm]
  \item  Derive the six explicit relations that express the
         $\ket{2;\ell,m}$ states in the Cartesian basis. (You may quote
         the raising‑operator identities from Problem Set 8.)\vv
  \item  From your result deduce the
         degeneracy formula $g_n = \dbinom{n+2}{2}$ for general $n$.
\end{enumerate}

% --------------------------------------------------------------------- %
\newpage
{\large\bfseries 3.\; Hydrogen atom — uncertainties}\qs

The ground‑state wave‑function of hydrogen is
$\displaystyle \psi_{100}(r)=\frac{1}{\sqrt{\pi a^3}}e^{-r/a}$.

\begin{enumerate}[label=\textbf{\alph*)}, leftmargin=1.2cm]
  \item  Compute $\expval{x}$, $\expval{x^2}$ and hence
         $\displaystyle\Delta x$.\vv
  \item  Using 
         $\hat p_x = -i\hbar\dv{}{x}$
         in Cartesian components,
         show that
         $\displaystyle\Delta p_x = \frac{\hbar}{a}$.\vv
  \item  Verify that the Heisenberg product
         $\Delta x\,\Delta p_x$ exceeds $\hbar/2$.
\end{enumerate}

% --------------------------------------------------------------------- %
\newpage
{\large\bfseries 4.\; Probability inside a sphere (excited hydrogen)}\qs

For the $n=2$, $\ell=1$ hydrogen states
\[
  \psi_{2;1,m}(r,\theta,\phi)
  = \frac{1}{(32\pi a^5)^{1/2}}\,
    r\,e^{-r/2a}\,Y_{1m}(\theta,\phi),
\]
define $P(b)$ to be the probability of finding the electron inside
$r\le b$.

\begin{enumerate}[label=\textbf{\alph*)}, leftmargin=1.2cm]
  \item  Show that
    $\displaystyle
      P(b)=1-e^{-b/a}\!\Bigl(1+\frac{b}{a}+\tfrac12\frac{b^{2}}{a^{2}}\Bigr)
      $ and comment on its $m$‑independence.\vv
  \item  Expand for small $b/a$ and confirm that
        $P(b)\approx \dfrac{1}{15}\bigl(b/a\bigr)^{5}$.
\end{enumerate}

% --------------------------------------------------------------------- %
\newpage
{\large\bfseries 5.\; Charged particle in $\,\mathbf B\!$ + axial trap}\qs

A particle of charge $q$ moves in $\mathbf B=B_0\hat z$ with vector
potential $\mathbf A=(0,B_0x,0)$ and is confined by
$V=\frac12m\omega_0^{2}z^{2}$.

\begin{enumerate}[label=\textbf{\alph*)}, leftmargin=1.2cm]
  \item  Using separation of variables and Landau‑level notation,
         write down the complete set of energy eigenvalues.\vv
  \item  State the degeneracy of each Landau level for a fixed box size
         $L_y$ (periodic boundary conditions).\vv
  \item  Briefly explain how the $z$‑motion modifies the density of
         states compared with the pure‑2D case.
\end{enumerate}

% --------------------------------------------------------------------- %
\newpage
{\large\bfseries 6.\; Spin‑$\tfrac32$ matrices}\qs

With the ordered basis
$\{\ket{+3/2},\ket{+1/2},\ket{-1/2},\ket{-3/2}\}$ take
\[
  \hat S_z = \frac{\hbar}{2}
    \begin{pmatrix} 3&0&0&0\\ 0&1&0&0\\ 0&0&-1&0\\ 0&0&0&-3 \end{pmatrix}.
\]

\begin{enumerate}[label=\textbf{\alph*)}, leftmargin=1.2cm]
  \item  Construct $\hat S_+$ and $\hat S_-$ via
        $\hat S_\pm=\hat S_x\pm i\hat S_y$ and give
        explicit $4\times4$ matrices.\vv
  \item  Hence write $\hat S_x$ and $\hat S_y$ and verify that they are
        Hermitian.\vv
  \item  Confirm the commutation relation
        $[\hat S_x,\hat S_y]=i\hbar\hat S_z$ in this representation.
\end{enumerate}

% --------------------------------------------------------------------- %
\newpage
{\large\bfseries 7.\; Spin–\,$\tfrac12$ in an infinite square well}\qs

Consider the Hamiltonian
$
 \hat H = \hat p^{2}/2m + V(x)\,\openone_2 + \beta\,\hat S_x
$
with $V(x)=0$ for $0<x<L$ and $V=\infty$ otherwise.

\begin{enumerate}[label=\textbf{\alph*)}, leftmargin=1.2cm]
  \item  Solve the time‑independent Schrödinger equation and give the
        energy eigenvalues.\vv
  \item  If the initial state is
        $
          \displaystyle
          \Psi(x,0)=\sqrt{\frac{2}{L}}
          \begin{pmatrix}
            \sin\!\bigl(\pi x/L\bigr)\\[4pt] 0
          \end{pmatrix},
          $
        find $\Psi(x,t)$.\vv
  \item  Determine the probability that a measurement of $\hat S_x$ at
        time $t$ yields $+\tfrac{\hbar}{2}$.
\end{enumerate}

% --------------------------------------------------------------------- %
\newpage
{\large\bfseries 8.\; Spin‑rotation operators}\qs

In the spin‑\,$\tfrac12$ representation $\hat S = \frac{\hbar}{2}\,\boldsymbol\sigma$.

\begin{enumerate}[label=\textbf{\alph*)}, leftmargin=1.2cm]
  \item  Show that
        $\displaystyle
          U_x(\alpha)=e^{-i\alpha\hat S_x/\hbar}
          =\cos\!\frac{\alpha}{2}\,\openone
          -i\sin\!\frac{\alpha}{2}\,\sigma_x,
        $
        and write the analogous formula for $U_z(\alpha)$.\vv
  \item  Verify the rotation‑axis identity
        $
          U_x(\pi/2)^{-1}\,U_y(\alpha)\,U_x(\pi/2)=U_z(\alpha).
        $
\end{enumerate}

% --------------------------------------------------------------------- %
\newpage
\begin{center}
  {\large\slshape End of mock Exam 2}\bigskip

  \emph{Solutions will be posted after the in‑class test.}
\end{center}

\end{document}
