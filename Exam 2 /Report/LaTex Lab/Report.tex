\documentclass[12pt]{article}

% ----------------------------------------------------------------------
% basic packages
\usepackage{amsmath, amssymb, physics}
\usepackage[margin=1in]{geometry}
\usepackage{enumitem}

% ----------------------------------------------------------------------
% paragraph spacing; no indent
\setlength{\parindent}{0pt}
\setlength{\parskip}{6pt}

% handy vertical‑space macros
\newcommand{\qs}{\bigskip\bigskip}
\newcommand{\vv}{\medskip}

% -------------------------------------------------------  document  ---
\begin{document}

% =========================================================================
\begin{center}
  {\Large\bfseries PHYSICS 373 \\
               Second Mid‑Term Examination}\\[2ex]
  Time limit: \textbf{1 hour} \quad (calculator permitted)
\end{center}

% --------------------------------------------------------------------- %
\newpage










%%%%%%%%%%%%%%%%%%%%%%%%%%%%%%%%%%%%%%%%%%%%%%%%%%%%%%%%%%%%
%  Q 1 – True / False (with short explanations)           %
%%%%%%%%%%%%%%%%%%%%%%%%%%%%%%%%%%%%%%%%%%%%%%%%%%%%%%%%%%%%
\newpage
\section*{Question 1 — True / False (circle one)}

\begin{enumerate}[label=\textbf{\alph*)},itemsep=1.2\baselineskip]

%----------------------------------------------------------
\item \textbf{For the 3‑D isotropic harmonic oscillator, the $n=2$ energy level is 6‑fold
      degenerate.}

      \textbf{Answer:} \textbf{T}\\
      \textbf{Why?}  In three dimensions the degeneracy is
      $\displaystyle g_n=\binom{n+2}{2}$.  
      For $n=2$ one obtains $g_2=\binom{4}{2}=6$.\par
      \emph{Concept cue:}  The integer $n=n_1+n_2+n_3$ can be “distributed’’ among
      the three Cartesian directions in $\binom{n+2}{2}$ ways (stars‑and‑bars
      combinatorics).

%----------------------------------------------------------
\item \textbf{The spherical harmonic $Y_{\ell,\ell}(\theta,\phi)$ is independent of $\theta$.}

      \textbf{Answer:} \textbf{F}\\
      \textbf{Why?}  $Y_{\ell,\ell}\propto\sin^\ell\!\theta\,e^{i\ell\phi}$;
      the $\sin^\ell\!\theta$ factor clearly depends on~$\theta$.\par
      \emph{Concept cue:}  Highest‑weight harmonics peak on the equator
      ($\theta=\pi/2$) and vanish at the poles, so some $\theta$‑dependence is
      unavoidable.

%----------------------------------------------------------
\item \textbf{Adding the operator $\dfrac{eB}{2m_e}\,\hat L_z$ to the hydrogen Hamiltonian
      removes \emph{all} degeneracy of the $n=2$ levels.}

      \textbf{Answer:} \textbf{F}\\
      \textbf{Why?}  The Zeeman term lifts only the $m$ (magnetic) degeneracy; the
      $\ell$‑degeneracy between $2s$ and $2p$ states
      ($\ell=0$ vs.\ $\ell=1$) survives.\par
      \emph{Concept cue:}  A perturbation $\propto\hat L_z$ shifts states with
      different $m$’s but leaves states that share the same $m$ unaffected,
      regardless of~$\ell$.

%----------------------------------------------------------
\item \textbf{For any Hermitian operator $\hat F$ and any quantum state
      $\ket{\psi}$, the product $\Delta F\,\Delta L_z$ can reach zero.}

      \textbf{Answer:} \textbf{F}\\
      \textbf{Why?}  Both uncertainties vanish only if $\ket{\psi}$ is a simultaneous
      eigenstate of \(\hat F\) and \(\hat L_z\), which requires
      $[\hat F,\hat L_z]=0$.  For a generic Hermitian $\hat F$ this commutator is
      non‑zero, so the product cannot be zero (Heisenberg
      inequality).\par
      \emph{Concept cue:}  Zero uncertainties in two observables are possible
      \emph{only} when the observables commute and the state lies in their common
      eigenbasis.

%----------------------------------------------------------
\item \textbf{The spin‑$\tfrac{3}{2}$ matrices can be chosen purely real in the
      $S_z$ eigen‑basis.}

      \textbf{Answer:} \textbf{F}\\
      \textbf{Why?}  While $\hat S_z$ is real and diagonal in that basis,
      at least one of $\hat S_x,\hat S_y$ must contain imaginary entries;
      in particular $\hat S_y$ is proportional to the anti‑Hermitian part
      $(\hat S_+-\hat S_-)$ and cannot be made real simultaneously with
      $\hat S_x$.\par
      \emph{Concept cue:}  The spin commutation relation
      $[\hat S_x,\hat S_y]=i\hbar\hat S_z$ forces an $i$ in \emph{some}
      matrix entries in any basis.

\end{enumerate}









% --------------------------------------------------------------------- %
%%%%%%%%%%%%%%%%%%%%%%%%%%%%%%%%%%%%%%%%%%%%%%%%%%%%%%%%%%%%
%  Q 2 – Exponential–weight differential operator          %
%%%%%%%%%%%%%%%%%%%%%%%%%%%%%%%%%%%%%%%%%%%%%%%%%%%%%%%%%%%%
\newpage
\section*{Question 2}

\textbf{Given}
\[
\hat O_{1}= i\,e^{-\beta x}\frac{\dd}{\dd x},
\qquad \beta\in\mathbb R,\qquad
\mathcal H=L^{2}(\mathbb R),
\]
answer the following:

\begin{enumerate}[label=\textbf{\alph*)},itemsep=1.5\baselineskip]

%----------------------------------------------------------
\item \textbf{Adjoint of $\hat O_{1}$.}

For $\phi,\psi\!\in\!\mathcal H$ we require
\[
\langle\phi|\hat O_{1}\psi\rangle=\langle\hat O_{1}^{\dagger}\phi|\psi\rangle
\quad\Longleftrightarrow\quad
\int_{-\infty}^{\infty}\! \phi^{*}\,
i\,e^{-\beta x}\psi'\,\dd x
= \int_{-\infty}^{\infty}\!\!
\bigl(\hat O_{1}^{\dagger}\phi\bigr)^{*}\psi\,\dd x.
\]

Integrate the LHS by parts (surface terms vanish for square–integrable
wave‑functions):

\begin{align*}
\int \phi^{*}i e^{-\beta x}\psi'\,\dd x
&= i\Bigl[\phi^{*}e^{-\beta x}\psi\Bigr]_{-\infty}^{\infty}
  - i\!\!\int\!\!\Bigl(\partial_{x}\phi^{*}\Bigr)e^{-\beta x}\psi\,\dd x
  + i\beta\!\!\int\!\!\phi^{*}e^{-\beta x}\psi\,\dd x \\[4pt]
&= -\,i\!\!\int\!e^{-\beta x}\Bigl(\partial_{x}\phi^{*}-\beta\phi^{*}\Bigr)\psi\,\dd x .
\end{align*}

Hence the adjoint operator must act as

\[
\boxed{\;
\hat O_{1}^{\dagger}
= -\,i\,e^{-\beta x}\Bigl(\partial_{x}-\beta\Bigr)
= -\,i\,e^{-\beta x}\frac{\dd}{\dd x}
      - i\beta\,e^{-\beta x}}\;.
\]

%----------------------------------------------------------
\item \textbf{Hermiticity of }$\displaystyle\hat O=\hat O_{1}+a\,e^{-\beta x}$.

\[
\hat O^{\dagger}
= \hat O_{1}^{\dagger}+a^{*}e^{-\beta x}
= \bigl[-\,i\,e^{-\beta x}\partial_{x}-i\beta e^{-\beta x}\bigr]
  + a^{*}e^{-\beta x}.
\]

For $\hat O$ to be Hermitian we need $\hat O^{\dagger}=\hat O$,
i.e.\ every term must match:

\[
\begin{cases}
\displaystyle +\,i\,e^{-\beta x}\partial_{x}\;=\;
           -\,i\,e^{-\beta x}\partial_{x}
           &\Longrightarrow\; \text{impossible for }\beta\neq 0,\\[6pt]
\displaystyle a\,e^{-\beta x}\;=\;-\,i\beta e^{-\beta x}+a^{*}e^{-\beta x}.
\end{cases}
\]

The first line already fails unless $\beta=0$.
\emph{Therefore, for any non‑zero real $\beta$ \underline{no choice of
complex $a$} can render $\hat O$ Hermitian}.  
(If $\beta=0$, $\hat O_{1}=i\partial_{x}$ is anti‑Hermitian
and one can offset it with a pure‑imaginary $a$; this is the trivial
$m=0$ case.)

%----------------------------------------------------------
\item \textbf{Expectation value $\langle\hat O\rangle$ in the harmonic‑oscillator
ground state $\psi_{0}(x)
        = (\alpha/\pi)^{1/4}\exp(-\tfrac12\alpha x^{2})$,
      with $\alpha=m\omega/\hbar$.}

Because $\hat O=\hat O_{1}+a e^{-\beta x}$,
\[
\langle\hat O\rangle
= i\int_{-\infty}^{\infty}\!e^{-\beta x}\psi_{0}\psi_{0}'\,\dd x
  + a\int_{-\infty}^{\infty}\!e^{-\beta x}\psi_{0}^{2}\,\dd x .
\]

\textit{Step 1: } rewrite the first integral via
$(\psi_{0}^{2})' = 2\psi_{0}\psi_{0}'$:
\[
i\int e^{-\beta x}\psi_{0}\psi_{0}'\,\dd x
= \frac{i}{2}\int e^{-\beta x}(\psi_{0}^{2})'\dd x
= \frac{i\beta}{2}\int e^{-\beta x}\psi_{0}^{2}\dd x ,
\]
where the boundary term vanishes.

\textit{Step 2: } evaluate the common Gaussian integral
\[
J(\beta)\equiv\int_{-\infty}^{\infty}
  e^{-\beta x}\,\psi_{0}^{2}(x)\,\dd x
= \sqrt{\frac{\alpha}{\pi}}\int_{-\infty}^{\infty}
     e^{-\alpha x^{2}-\beta x}\,\dd x
= \exp\!\Bigl(\frac{\beta^{2}}{4\alpha}\Bigr).
\]

\textbf{Result:}
\[
\boxed{\;
\langle\hat O\rangle
= \Bigl[\frac{i\beta}{2}+a\Bigr]\,
   \exp\!\Bigl(\frac{\beta^{2}}{4\alpha}\Bigr)}.
\]

\emph{Concept cues.}
\begin{itemize}
\item Adjoint of a differential operator: integrate by parts, mind the weight
      $e^{-\beta x}$.
\item Hermiticity demands equality of the full operators, not just of their
      matrix elements.
\item Gaussian integrals with a linear term are handled by completing the
      square.
\end{itemize}

\end{enumerate}















%%%%%%%%%%%%%%%%%%%%%%%%%%%%%%%%%%%%%%%%%%%%%%%%%%%%%%%%%%%%
%  Q 3 – Two–state system                                  %
%%%%%%%%%%%%%%%%%%%%%%%%%%%%%%%%%%%%%%%%%%%%%%%%%%%%%%%%%%%%
\newpage
\section*{Question 3 — Two–state system}

\[
    \hat H = E_{0}
    \begin{pmatrix}
       2 & 1+i\\
       * & 3
    \end{pmatrix},\qquad
    \ket{\psi}=\begin{pmatrix}0\\ 1\end{pmatrix}.
\]

\begin{enumerate}[label=\textbf{\alph*)},itemsep=1.5\baselineskip]

%----------------------------------------------------------
\item \textbf{Missing matrix element.}

$\hat H$ must be Hermitian, i.e.\ $H_{21}=H_{12}^{*}$.
Because $H_{12}=1+i$,
\[
\boxed{\,* = 1-i\,}.
\]

%----------------------------------------------------------
\item \textbf{Expectation value and uncertainty in the state $\ket{\psi}$.}

With the now–Hermitian matrix
\(
M = \tfrac1{E_{0}}\hat H=
\begin{psmallmatrix}
2 & 1+i\\ 1-i & 3
\end{psmallmatrix},
\)
\[
\langle H\rangle
= \bra{\psi}\hat H\ket{\psi}
= E_{0}\,(0,1)\!
    \begin{psmallmatrix} 2&1+i\\1-i&3\end{psmallmatrix}
    \begin{psmallmatrix}0\\1\end{psmallmatrix}
= 3E_{0}.
\]

To find the variance we need $\langle H^{2}\rangle$.
A short matrix product gives
\[
M^{2}=
\begin{pmatrix}
 6 & 5(1+i)\\ 5(1-i) & 11
\end{pmatrix}
\Longrightarrow
\langle H^{2}\rangle = E_{0}^{2}\,(0,1)
  \begin{psmallmatrix}6&5(1+i)\\5(1-i)&11\end{psmallmatrix}
  \begin{psmallmatrix}0\\1\end{psmallmatrix}
= 11E_{0}^{2}.
\]

\[
\Delta H=\sqrt{\langle H^{2}\rangle-\langle H\rangle^{2}}
        =\sqrt{11E_{0}^{2}-9E_{0}^{2}}
        =E_{0}\sqrt{2}.
\]

\[
\boxed{\,
   \langle H\rangle = 3E_{0},\qquad
   \Delta H = E_{0}\sqrt{2}\,}.
\]

%----------------------------------------------------------
\item \textbf{Possible measured energies (eigenvalues of $\hat H$).}

Solve $\det(M-\lambda I)=0$:
\[
\lambda^{2}-5\lambda+4=0
\;\Longrightarrow\;
\lambda_{1}=1,\quad
\lambda_{2}=4,
\]
so the physical energies are
\[
\boxed{E_{1}=E_{0},\qquad E_{2}=4E_{0}}.
\]

%----------------------------------------------------------
\item \textbf{Probability of measuring $E_{0}$ in the state $\ket{\psi}$.}

Eigenvector for $\lambda_{1}=1$:

\(
(M-I)
\begin{psmallmatrix}v_{1}\\v_{2}\end{psmallmatrix}=0
\Longrightarrow
v_{1}=-(1+i)v_{2}.
\)

Choose $v_{2}=1$; normalise:

\(
\norm{v}^{2}=(1+i)(1-i)+1=3
\Longrightarrow
\ket{E_{0}}=\frac{1}{\sqrt3}
  \begin{psmallmatrix}-(1+i)\\1\end{psmallmatrix}.
\)

Overlap with $\ket{\psi}$:
\(
\braket{E_{0}}{\psi}=1/\sqrt3,
\)
so

\[
\boxed{\;
  P(E_{0}) = \bigl|\braket{E_{0}}{\psi}\bigr|^{2}= \frac{1}{3}\; }.
\]

\emph{Check:} $P(4E_{0})=1-P(E_{0})=\tfrac23$, confirming probability
conservation.

\end{enumerate}















%=============================================================
\newpage
\section*{4.\; Probability inside a sphere (excited hydrogen)}

\textbf{For the $n=2,\;\ell =1$ hydrogen states}
\[
\psi_{2;1,m}(r,\theta,\phi)=
\frac{1}{\!\bigl(32\pi a^{5}\bigr)^{1/2}}\;
r\,e^{-\,r/2a}\,Y_{1m}(\theta,\phi),
\qquad 
a=\frac{\hbar^{2}}{m_{e}e^{2}}
\]
\textbf{define $P(b)$ to be the probability of finding the electron inside
$r\le b$.}

\begin{enumerate}[label=\textbf{\alph*)},leftmargin=0.9cm]

%-----------------------------------------------------------------
\item \textbf{Show that}
\[
P(b)=1-e^{-\,b/a}\Bigl(1+\tfrac{b}{a}+\tfrac12\,\tfrac{b^{2}}{a^{2}}\Bigr),
\]
\textbf{and comment on its $m$–independence.}

\medskip
\underline{\emph{Solution}}

The probability is a radial integral because
\(\int|Y_{1m}|^{2}d\Omega=1\) for \emph{any} $m=-1,0,1$:
\[
P(b)=\int_{0}^{b}\!dr\,r^{2}\!\int d\Omega\;
\bigl|\psi_{2;1,m}(r,\theta,\phi)\bigr|^{2}.
\]
With \(|\psi|^{2}=\dfrac{1}{32\pi a^{5}}\,
r^{2}e^{-r/a}\,|Y_{1m}|^{2}\) we get
\[
P(b)=\frac{4\pi}{32\pi a^{5}}\int_{0}^{b}\! r^{4}\,e^{-\,r/a}\,dr
      =\frac{1}{8a^{5}}\int_{0}^{b} r^{4}e^{-\,r/a}\,dr.
\]
Use the standard result
\(
\displaystyle\int r^{n}e^{-r/a}dr
        =-a\,r^{n}e^{-r/a}-an\,\int r^{\,n-1}e^{-r/a}dr
\)
or look up the incomplete Gamma‑function; performing the integral gives
\[
\int_{0}^{b}\! r^{4}e^{-\,r/a}dr
   =a^{5}\!\left[1-e^{-\,b/a}\Bigl(1+\tfrac{b}{a}
                                    +\tfrac12\tfrac{b^{2}}{a^{2}}
                                    +\tfrac16\tfrac{b^{3}}{a^{3}}
                                    +\tfrac1{24}\tfrac{b^{4}}{a^{4}}\Bigr)\right].
\]
Multiplying by $1/(8a^{5})$ leaves only the first three terms inside the
parentheses, because the $b^{3}$ and $b^{4}$ pieces cancel:
\[
P(b)=1-e^{-\,b/a}\Bigl(1+\tfrac{b}{a}+\tfrac12\tfrac{b^{2}}{a^{2}}\Bigr).
\]

\emph{Independence of $m$.}  All three $m$–states share the same radial
factor \(r\,e^{-r/2a}\); the angular integral \(\int|Y_{1m}|^{2}d\Omega=1\) is
$m$‑independent, so \(P(b)\) is identical for \(m=-1,0,+1\).

%-----------------------------------------------------------------
\item \textbf{Expand for small $b/a$ and confirm that
\(\displaystyle P(b)\approx \tfrac1{15}(b/a)^{5}\).  What is $c$?}

\medskip
\underline{\emph{Solution}}

For \(x=b/a\ll1\) write
\(e^{-x}=1-x+\tfrac{x^{2}}{2}-\tfrac{x^{3}}{6}
        +\tfrac{x^{4}}{24}-\tfrac{x^{5}}{120}+O(x^{6})\).
Insert this into part (a):
\[
P(b)=1-\Bigl(1-x+\tfrac{x^{2}}{2}-\tfrac{x^{3}}{6}
              +\tfrac{x^{4}}{24}-\tfrac{x^{5}}{120}+\ldots\Bigr)
           \Bigl(1+x+\tfrac{x^{2}}{2}\Bigr).
\]
Expand and keep terms up to \(x^{5}\):
\[
P(b)=1-\Bigl(1+x+\tfrac{x^{2}}{2}
             -x-\tfrac{x^{2}}{2}-\tfrac{x^{3}}{6}
             +\tfrac{x^{2}}{2}+\tfrac{x^{3}}{2}+\tfrac{x^{4}}{4}
             -\tfrac{x^{3}}{6}-\tfrac{x^{4}}{6}-\tfrac{x^{5}}{12}\Bigr)
     =\frac{x^{5}}{15}+O(x^{6}).
\]
Thus for \(b\ll a\)
\[
\boxed{P(b)\simeq c\left(\frac{b}{a}\right)^{5}},
\qquad c=\frac{1}{15}.
\]

\end{enumerate}










% --------------------------------------------------------------------- %
\newpage
{\large\bfseries 5.\; Charged particle in $\,\mathbf B\!$ + axial trap}\qs

A particle of charge $q$ moves in $\mathbf B=B_0\hat z$ with vector
potential $\mathbf A=(0,B_0x,0)$ and is confined by
$V=\frac12m\omega_0^{2}z^{2}$.

\begin{enumerate}[label=\textbf{\alph*)}, leftmargin=1.2cm]
  \item  Using separation of variables and Landau‑level notation,
         write down the complete set of energy eigenvalues.\vv
  \item  State the degeneracy of each Landau level for a fixed box size
         $L_y$ (periodic boundary conditions).\vv
  \item  Briefly explain how the $z$‑motion modifies the density of
         states compared with the pure‑2D case.
\end{enumerate}






% --------------------------------------------------------------------- %
\newpage
{\large\bfseries 6.\; Spin‑$\tfrac32$ matrices}\qs

With the ordered basis
$\{\ket{+3/2},\ket{+1/2},\ket{-1/2},\ket{-3/2}\}$ take
\[
  \hat S_z = \frac{\hbar}{2}
    \begin{pmatrix} 3&0&0&0\\ 0&1&0&0\\ 0&0&-1&0\\ 0&0&0&-3 \end{pmatrix}.
\]

\begin{enumerate}[label=\textbf{\alph*)}, leftmargin=1.2cm]
  \item  Construct $\hat S_+$ and $\hat S_-$ via
        $\hat S_\pm=\hat S_x\pm i\hat S_y$ and give
        explicit $4\times4$ matrices.\vv
  \item  Hence write $\hat S_x$ and $\hat S_y$ and verify that they are
        Hermitian.\vv
  \item  Confirm the commutation relation
        $[\hat S_x,\hat S_y]=i\hbar\hat S_z$ in this representation.
\end{enumerate}






% --------------------------------------------------------------------- %
\newpage
{\large\bfseries 7.\; Spin–\,$\tfrac12$ in an infinite square well}\qs

Consider the Hamiltonian
$
 \hat H = \hat p^{2}/2m + V(x)\,\openone_2 + \beta\,\hat S_x
$
with $V(x)=0$ for $0<x<L$ and $V=\infty$ otherwise.

\begin{enumerate}[label=\textbf{\alph*)}, leftmargin=1.2cm]
  \item  Solve the time‑independent Schrödinger equation and give the
        energy eigenvalues.\vv
  \item  If the initial state is
        $
          \displaystyle
          \Psi(x,0)=\sqrt{\frac{2}{L}}
          \begin{pmatrix}
            \sin\!\bigl(\pi x/L\bigr)\\[4pt] 0
          \end{pmatrix},
          $
        find $\Psi(x,t)$.\vv
  \item  Determine the probability that a measurement of $\hat S_x$ at
        time $t$ yields $+\tfrac{\hbar}{2}$.
\end{enumerate}

% --------------------------------------------------------------------- %
\newpage
{\large\bfseries 8.\; Spin‑rotation operators}\qs

In the spin‑\,$\tfrac12$ representation $\hat S = \frac{\hbar}{2}\,\boldsymbol\sigma$.

\begin{enumerate}[label=\textbf{\alph*)}, leftmargin=1.2cm]
  \item  Show that
        $\displaystyle
          U_x(\alpha)=e^{-i\alpha\hat S_x/\hbar}
          =\cos\!\frac{\alpha}{2}\,\openone
          -i\sin\!\frac{\alpha}{2}\,\sigma_x,
        $
        and write the analogous formula for $U_z(\alpha)$.\vv
  \item  Verify the rotation‑axis identity
        $
          U_x(\pi/2)^{-1}\,U_y(\alpha)\,U_x(\pi/2)=U_z(\alpha).
        $
\end{enumerate}

% --------------------------------------------------------------------- %
\newpage
\begin{center}
  {\large\slshape End of mock Exam 2}\bigskip

  \emph{Solutions will be posted after the in‑class test.}
\end{center}

\end{document}
