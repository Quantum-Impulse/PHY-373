\documentclass[12pt]{article}

% -------------------------------------------------
% Packages
% -------------------------------------------------
\usepackage{amsmath, amssymb, amsthm, physics} % math & bra–ket notation
\usepackage{geometry}      % page margins
\usepackage{graphicx}      % figures (if needed)
\usepackage{enumitem}      % custom lists
\usepackage{hyperref}      % hyperlinks
\usepackage{fancyhdr}      % header / footer

% -------------------------------------------------
% Page layout
% -------------------------------------------------
\geometry{margin=1in}
\setlength{\parindent}{0pt}
\setlength{\parskip}{8pt}

% -------------------------------------------------
% Header / footer
% -------------------------------------------------
\pagestyle{fancy}
\fancyhead[L]{PHYSICS 373 – Problem Set 11}
\fancyhead[R]{Enrique Rivera Jr}
\fancyfoot[C]{\thepage}

% -------------------------------------------------
% Theorem‑like environments (if you need them)
% -------------------------------------------------
\newtheorem{theorem}{Theorem}
\newtheorem{lemma}{Lemma}
\newtheorem{corollary}{Corollary}
\newtheorem{proposition}{Proposition}

\begin{document}

\title{Problem Set 11 Solutions}
\author{Enrique Rivera Jr}
\date{\today}

\maketitle

% -------------------------------------------------
% Question 1
% -------------------------------------------------
\section*{Question 1: Charged Particle in \(\mathbf B=B_0\hat z\) with Axial Trap}
We work in Landau gauge
\(\mathbf A=(0,\,xB_0,\,0)\) so that
\(\mathbf B=\nabla\times\mathbf A=B_0\hat z\).  The Hamiltonian is
\begin{equation}
 H\;=\;\frac{1}{2m}\bigl(\mathbf p-q\mathbf A\bigr)^2 
            +\frac12 m\omega_0^{\,2}z^2.
\end{equation}

---
\subsection*{1. Separation of variables}
Impose the periodic boundary condition in~$y$:
\(\psi(x,y,z)=\psi(x,y+L,z)\).  Write a separable ansatz
\begin{equation}
 \psi(x,y,z)=e^{ik_y y}\;\chi_n(x)\;\zeta_{n_z}(z),
 \qquad k_y=\frac{2\pi N}{L},\;N\in\mathbb Z.
\end{equation}
The momenta become
\(p_y\to\hbar k_y\) and \(p_x=-i\hbar\partial_x\).

---
\subsection*{2.  $x$‐motion \,(Landau oscillator)}
\begin{align}
 H_x &=\frac1{2m}\Bigl[p_x^2+\bigl(p_y-qB_0x\bigr)^2\Bigr]
     =\frac1{2m}\Bigl[p_x^2+m^2\omega_c^{2}\bigl(x-x_0\bigr)^2\Bigr],\\[2pt]
 \omega_c &\equiv \frac{|q|B_0}{m}, \qquad x_0\equiv \frac{\hbar k_y}{qB_0}.
\end{align}
This is a 1‑D harmonic oscillator of frequency \(\omega_c\).  Its eigen‑energies are
\begin{equation}
 E_x\;=\;\hbar\omega_c\Bigl(n+\tfrac12\Bigr),\qquad n=0,1,2,\dots
\label{landau}
\end{equation}
independent of \(k_y\): \emph{Landau levels}.

---
\subsection*{3.  $z$‐motion \,(axial harmonic trap)}
\begin{equation}
 H_z = \frac{p_z^{2}}{2m}+\frac12 m\omega_0^{\,2}z^{2}
 \quad\Longrightarrow\quad
 E_z = \hbar\omega_0\Bigl(n_z+\tfrac12\Bigr),\qquad n_z=0,1,2,\dots
\label{zlevels}
\end{equation}

---
\subsection*{4.  Total energy spectrum}
Add \eqref{landau} and \eqref{zlevels}:
\begin{equation}
\boxed{\rule{0pt}{12pt}
 E_{n,n_z}=\hbar\omega_c\Bigl(n+\tfrac12\Bigr)+\hbar\omega_0\Bigl(n_z+\tfrac12\Bigr),
 \qquad n,n_z\in\mathbb N_0.}
\end{equation}
Each Landau index~$n$ retains a macroscopic degeneracy labelled by the discrete
\(k_y=2\pi N/L\) (guiding‑center position $x_0$).  Per unit area, the degeneracy is
\(g=\tfrac{|q|B_0}{h}\).

\bigskip






% -------------------------------------------------
% Question 2
% -------------------------------------------------
\section*{Question 2: $4\!\times\!4$ Spin--$\tfrac32$ Matrices (Detailed Derivation)}

We work in the $|s,m\rangle$ basis with $s=\tfrac32$ and
\(m=+\tfrac32,\,+\tfrac12,\,-\tfrac12,\,-\tfrac32\) listed in that order.

---
\subsection*{1.  Define $\hat S_z$ and ladder matrices}
Set
\[
  \hat S_z=\frac{\hbar}{2}
  \begin{pmatrix}
    3&0&0&0\\ 0&1&0&0\\ 0&0&-1&0\\0&0&0&-3
  \end{pmatrix}.
\]
Using
\( \hat S_\pm|s,m\rangle = \hbar\sqrt{s(s+1)-m(m\pm1)}\,|s,m\pm1\rangle\),
non‑zero matrix elements are
\[
  \langle m+1|\hat S_+|m\rangle  = \hbar\sqrt{(\tfrac32-m)(\tfrac32+m+1)}.
\]
Numerically:
\begin{align*}
  &|{\tfrac32}\rangle\xrightarrow{S_+}0,\;\;|{\tfrac12}\rangle\xrightarrow{S_+}\hbar\sqrt3|{\tfrac32}\rangle,\\[2pt]
  &|{-\tfrac12}\rangle\xrightarrow{S_+}2\hbar|{\tfrac12}\rangle,\quad
  |{-\tfrac32}\rangle\xrightarrow{S_+}\hbar\sqrt3|{-\tfrac12}\rangle.
\end{align*}

Thus
\[
  \boxed{
    \hat S_+ = \hbar
      \begin{pmatrix}
        0 & \sqrt{3} & 0 & 0 \\
        0 & 0 & 2 & 0 \\
        0 & 0 & 0 & \sqrt{3} \\
        0 & 0 & 0 & 0
      \end{pmatrix}
  },\qquad
  \hat S_- = \hat S_+^{\dagger}.
\]

---
\subsection*{2.  Build $\hat S_x$ and $\hat S_y$}
\[
  \hat S_x = \frac{\hat S_+ + \hat S_-}{2},\qquad
  \hat S_y = \frac{\hat S_+ - \hat S_-}{2i}.
\]
Explicitly
\begin{align}
  \hat S_x &= \frac{\hbar}{2}
  \begin{pmatrix}
    0 & \sqrt3 & 0 & 0 \\
    \sqrt3 & 0 & 2 & 0 \\
    0 & 2 & 0 & \sqrt3 \\
    0 & 0 & \sqrt3 & 0
  \end{pmatrix},\\[6pt]
  \hat S_y &= \frac{\hbar}{2}
  \begin{pmatrix}
    0 & -i\sqrt3 & 0 & 0 \\
    i\sqrt3 & 0 & -2i & 0 \\
    0 & 2i & 0 & -i\sqrt3 \\
    0 & 0 & i\sqrt3 & 0
  \end{pmatrix}.
\end{align}

---
\subsection*{3.  Consistency checks}
\begin{itemize}[leftmargin=2.2em]
  \item **Hermiticity:** $\hat S_x^{\dagger}=\hat S_x$ and $\hat S_y^{\dagger}=\hat S_y$ follow from $\hat S_-^{\dagger}=\hat S_+$.  
  \item **$SU(2)$ algebra:** one may verify
  $[\hat S_x,\hat S_y]=i\hbar\hat S_z$ (and cyclic permutations) by direct multiplication.
  \item **Casimir:** $\hat S^{2}=\hat S_x^{2}+\hat S_y^{2}+\hat S_z^{2}=\hbar^{2}s(s+1)=\tfrac{15}{4}\hbar^{2}$ times the identity.
\end{itemize}
These matrices therefore furnish an explicit 4‑dimensional, spin‑$\tfrac32$ representation of \(\mathfrak{su}(2)\).

\bigskip



% -------------------------------------------------
% Question 3
% -------------------------------------------------
%------------------------------------------------
\section*{Question 3 \\ Spin–\(\tfrac12\) Particle in a 1‑D Infinite Square Well with a Transverse Field}

The Hamiltonian inside the well \((0<x<L)\) is
\[
  \hat H \;=\;
  \frac{\hat p^{2}}{2m}\;+\;\beta\,\hat S_x,
  \qquad
  \hat S_x=\frac{\hbar}{2}\sigma_x,
\]
while \(\Psi(0)=\Psi(L)=0\) enforces the usual infinite–wall boundary
conditions.  Outside the well the wave‑function vanishes identically.

%------------------------------------------------
\subsection*{(a) Stationary states}

\paragraph{1.  Decouple the spin components.}
Write the spinor in the \(S_z\) basis,
\(
  \Psi(x)=\!
    \begin{pmatrix}\psi_\uparrow(x)\\[2pt]\psi_\downarrow(x)\end{pmatrix},
\)
so that
\(
  \hat S_x\Psi
  =\frac{\hbar}{2}
    \begin{pmatrix}\psi_\downarrow \\ \psi_\uparrow\end{pmatrix}.
\)
The TISE \(\hat H\Psi = E\Psi\) gives two coupled equations:
\[
  -\frac{\hbar^{2}}{2m}\psi_\uparrow'' + \frac{\beta\hbar}{2}\,\psi_\downarrow
  = E\,\psi_\uparrow,
  \qquad
  -\frac{\hbar^{2}}{2m}\psi_\downarrow'' + \frac{\beta\hbar}{2}\,\psi_\uparrow
  = E\,\psi_\downarrow.
\]

\paragraph{2.  Switch to the \(\hat S_x\) eigen‑basis.}
Define
\[
  \phi_{+}(x)=\frac{\psi_\uparrow+\psi_\downarrow}{\sqrt2},\qquad
  \phi_{-}(x)=\frac{\psi_\uparrow-\psi_\downarrow}{\sqrt2},
\]
which correspond to spin eigenkets
\(\ket{+_x},\ket{-_x}\):
\(
  \Psi = \phi_{+}\ket{+_x}+\phi_{-}\ket{-_x}.
\)

In this basis the equations decouple:
\[
  -\frac{\hbar^{2}}{2m}\phi_{\pm}'' \;=\; (E\mp\tfrac{\beta\hbar}{2})\;\phi_{\pm}.
\]
Hence each \(\phi_{\pm}\) is simply a stationary level of the ordinary
square well with an energy offset \(\pm\tfrac{\beta\hbar}{2}\).

\paragraph{3.  Solutions and energy spectrum.}
Imposing \(\phi_{\pm}(0)=\phi_{\pm}(L)=0\) gives
\[
  \phi_{\pm,n}(x)=\sqrt{\frac{2}{L}}\,\sin\!\Bigl(\!\tfrac{n\pi x}{L}\Bigr),
  \qquad
  \boxed{\,E_{n,\pm}
   =\frac{\hbar^{2}\pi^{2}n^{2}}{2mL^{2}}
     \;\pm\;\frac{\beta\hbar}{2}},\quad n=1,2,\dots
\]
and the normalised stationary spinors
\[
  \Psi_{n,\pm}(x)=
    \sqrt{\frac{2}{L}}\,\sin\!\Bigl(\!\tfrac{n\pi x}{L}\Bigr)\ket{\pm_x}.
\]

%------------------------------------------------
\subsection*{(b) Time evolution of the specified initial state}

The problem gives
\[
  \Psi(x,0)=
    \sqrt{\frac{2}{L}}\,
    \sin\!\bigl(\tfrac{\pi x}{L}\bigr)\,
    \ket{\uparrow_z}
  =\frac{\phi_{+,1}(x)}{\sqrt2}\ket{+_x}
  +\frac{\phi_{-,1}(x)}{\sqrt2}\ket{-_x}.
\]

Attach phase factors \(e^{-iE_{1,\pm}t/\hbar}\):
\[
  \Psi(x,t)=
    \sqrt{\frac{2}{L}}\,
    \sin\!\bigl(\tfrac{\pi x}{L}\bigr)
    \frac{
      e^{-i(E_1^{(0)}+\Omega\hbar)\,t/\hbar}\ket{+_x}
      +e^{-i(E_1^{(0)}-\Omega\hbar)\,t/\hbar}\ket{-_x}
    }{\sqrt2},
  \qquad
  E_1^{(0)}=\frac{\hbar^{2}\pi^{2}}{2mL^{2}},\;
  \Omega=\frac{\beta}{2}.
\]

Re‑expressing \(\ket{\pm_x}\) back in the \(S_z\) basis
(\(\ket{\pm_x}=\tfrac{1}{\sqrt2}(\ket{\uparrow_z}\pm\ket{\downarrow_z})\))
gives the compact spinor form

\[
  \boxed{
    \Psi(x,t)=
      \sqrt{\frac{2}{L}}\,
      \sin\!\bigl(\tfrac{\pi x}{L}\bigr)
      \begin{pmatrix}
         \cos(\Omega t) \\
        -i\,\sin(\Omega t)
      \end{pmatrix}_{\!z}
    },
    \quad
    \Omega=\frac{\beta}{2}.
\]

\emph{Interpretation}: the spatial profile remains the ground standing‑wave
of the well, while the spin precesses about the \(x\)-axis with Rabi
frequency \(\Omega\) (Larmor precession in a static transverse field).




% -------------------------------------------------
% Question 4
% -------------------------------------------------
\section*{Question 4: Spin‑\(\tfrac12\) Rotation Operators}

Throughout we employ the Pauli matrices
\(\sigma_x=\begin{pmatrix}0&1\\1&0\end{pmatrix}\),
\(\sigma_y=\begin{pmatrix}0&-i\\ i&0\end{pmatrix}\),
\(\sigma_z=\begin{pmatrix}1&0\\0&-1\end{pmatrix}\),
with \(\hat S_i=\dfrac{\hbar}{2}\sigma_i\,(i=x,y,z).\)

%------------------------------------------------
\subsection*{4(a) Unitary rotation matrices \(U_x(\alpha),\,U_z(\alpha)\)}

A rotation through an angle \(\alpha\) about axis \(i\) is
\[
   U_i(\alpha)\;=\;
   \exp\!\Bigl(-\frac{i}{\hbar}\,\alpha\hat S_i\Bigr)
   \;=\;
   \exp\!\Bigl(-\tfrac{i}{2}\,\alpha\,\sigma_i\Bigr).
\]

Using \(\sigma_i^{2}=I\) one expands
\[
   e^{-i(\alpha/2)\sigma_i}
      \;=\;
      \cos\!\Bigl(\tfrac{\alpha}{2}\Bigr)\,I
      \;-\;
      i\,\sin\!\Bigl(\tfrac{\alpha}{2}\Bigr)\,\sigma_i,
\]
so that
\[
   \boxed{U_x(\alpha)=
          \cos\!\tfrac{\alpha}{2}\,I
          -i\sin\!\tfrac{\alpha}{2}\,\sigma_x},
   \qquad
   \boxed{U_z(\alpha)=
          \cos\!\tfrac{\alpha}{2}\,I
          -i\sin\!\tfrac{\alpha}{2}\,\sigma_z}.
\]

%------------------------------------------------
\subsection*{4(b) Rotation–axis equivalence}

We already know
\(
   U_y(\alpha)=
   \cos\!\tfrac{\alpha}{2}\,I
   -i\sin\!\tfrac{\alpha}{2}\,\sigma_y .
\)
Take the special rotation about \(x\)
\[
  U_x\!\Bigl(\tfrac{\pi}{2}\Bigr)
  =\exp\!\Bigl(-\tfrac{i\pi}{4}\sigma_x\Bigr)
  =\frac{1}{\sqrt2}\bigl(I-i\sigma_x\bigr),
  \;\;
  U_x\!\Bigl(-\tfrac{\pi}{2}\Bigr)=U_x(\tfrac{\pi}{2})^{\dagger}
  =\frac{1}{\sqrt2}\bigl(I+i\sigma_x\bigr).
\]

Conjugating \(U_y(\alpha)\) gives
\[
  \begin{aligned}
  U_x\!\Bigl(\tfrac{\pi}{2}\Bigr)^{-1}
  U_y(\alpha)
  U_x\!\Bigl(\tfrac{\pi}{2}\Bigr)
  &=
  \cos\!\tfrac{\alpha}{2}\,I
  -i\sin\!\tfrac{\alpha}{2}\,
    \Bigl[
       U_x\!\Bigl(-\tfrac{\pi}{2}\Bigr)\,
       \sigma_y\,
       U_x\!\Bigl(\tfrac{\pi}{2}\Bigr)
    \Bigr]\\[4pt]
  &=\cos\!\tfrac{\alpha}{2}\,I
    -i\sin\!\tfrac{\alpha}{2}\,\sigma_z
    \;=\;
    U_z(\alpha),
  \end{aligned}
\]
because a \(\pi/2\) rotation about \(x\) maps \(\sigma_y\!\to\!\sigma_z\).
Hence
\[
   \boxed{\,U_x(\pi/2)^{-1}\,U_y(\alpha)\,U_x(\pi/2)=U_z(\alpha)\,},
\]
as required for the consistency of spin rotations.

%------------------------------------------------


\end{document}
