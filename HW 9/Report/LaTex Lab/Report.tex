\documentclass[12pt]{article}

% Packages for formatting
\usepackage{amsmath, amssymb, amsthm} % Math symbols & environments
\usepackage{geometry} % Page layout
\usepackage{graphicx} % Images (if needed)
\usepackage{enumitem} % Custom numbering
\usepackage{hyperref} % Hyperlinks
\usepackage{fancyhdr} % Header/Footer formatting
\usepackage{physics} % Includes \ket and \bra notation

% Page Formatting
\geometry{margin=1in} % 1-inch margins
\setlength{\parindent}{0pt} % No paragraph indent
\setlength{\parskip}{8pt} % Spacing between paragraphs

% Custom Header/Footer
\pagestyle{fancy}
\fancyhead[L]{PHYSICS 373 - Problem Set 9}
\fancyhead[R]{Enrique Rivera Jr}
\fancyfoot[C]{\thepage}

% Theorem Styles
\newtheorem{theorem}{Theorem}
\newtheorem{lemma}{Lemma}
\newtheorem{corollary}{Corollary}
\newtheorem{proposition}{Proposition}

\begin{document}

\title{Problem Set 9 Solutions}
\author{Enrique Rivera Jr}
\date{\today}

\maketitle


\section*{Question 1: Applying $L_{-}$ to $Y_{2,2}$}

\subsection*{1. Highest-weight Spherical Harmonic $Y_{2,2}$}
We have, in general,
\begin{equation}
Y_{l,l}(\theta,\phi) = A_l\, e^{i l \phi}\, \sin^l(\theta), \quad
A_l = (-1)^l \sqrt{\frac{(2l+1)!}{4\pi}}\,\frac{1}{2^l l!}.
\end{equation}
For $l=2$:
\begin{align*}
Y_{2,2}(\theta,\phi) &= A_2\, e^{2i\phi}\, \sin^2\theta, \\
A_2 &= (-1)^2 \sqrt{\frac{5!}{4\pi}}\,\frac{1}{2^2\,2!}
= \sqrt{\frac{120}{4\pi}}\,\frac{1}{4\cdot 2}
= \frac{\sqrt{30/\pi}}{8}.
\end{align*}
So explicitly:
\[
\boxed{\,Y_{2,2}(\theta,\phi) = \frac{\sqrt{30/\pi}}{8}\, e^{2i\phi}\, \sin^2\theta.}\]

\subsection*{2. The Lowering Operator $L_{-}$}
We recall:
\begin{equation}
L_{-} Y_{l,m} = -\,\hbar\,\sqrt{l(l+1) - m(m-1)}\; Y_{l,m-1}.
\end{equation}
In the problem statement:
\begin{equation}
L_{\pm} = \mp\,\hbar\, e^{\pm i\phi}\Bigl( \frac{\partial}{\partial \theta} \pm i\cot\theta\,\frac{\partial}{\partial \phi}\Bigr).
\end{equation}
We apply $L_{-}$ successively to $Y_{2,2}$.

\subsection*{3. Successive Applications of $L_{-}$}
\paragraph{(a) $n=1$: $m=2\to1$.}
\begin{align*}
L_{-} Y_{2,2} &= -\,\hbar \sqrt{2(2+1) - 2(2-1)}\,Y_{2,1}
= -\,\hbar \sqrt{6 - 2}\,Y_{2,1}
= -\,\hbar\cdot 2\,Y_{2,1}
= -2\hbar\,Y_{2,1}.
\end{align*}
\paragraph{(b) $n=2$: $m=1\to0$.}
\(
L_{-}^2 Y_{2,2} = L_{-}\bigl(L_{-}Y_{2,2}\bigr) = -2\hbar\,L_{-}Y_{2,1}.
\)
But:
\begin{align*}
L_{-}Y_{2,1} &= -\,\hbar\,\sqrt{6 - 1(1-1)}\,Y_{2,0}
= -\,\hbar\,\sqrt{6}\,Y_{2,0}.
\end{align*}
Hence,
\(
L_{-}^2 Y_{2,2} = -2\hbar\bigl( -\hbar\sqrt{6}\bigr)Y_{2,0} = +2\hbar^2\,\sqrt{6}\,Y_{2,0}.
\)

\paragraph{(c) $n=3$: $m=0\to-1$.}
\begin{align*}
L_{-}^3 Y_{2,2} &= L_{-}\bigl(L_{-}^2 Y_{2,2}\bigr)
= 2\hbar^2\,\sqrt{6}\;\bigl(L_{-}Y_{2,0}\bigr).\\
L_{-}Y_{2,0} &= -\,\hbar\,\sqrt{6}\,Y_{2,-1}\quad(\text{since }m=0,\,m(m-1)=0\cdot(-1)=0,\,2(2+1)=6).\
&= -\,\hbar\,\sqrt{6}\,Y_{2,-1}.
\end{align*}
Thus,
\(
L_{-}^3 Y_{2,2} = 2\hbar^2\sqrt{6}\bigl(-\hbar\sqrt{6}\bigr)\,Y_{2,-1} = -12\hbar^3\,Y_{2,-1}.
\)

\paragraph{(d) $n=4$: $m=-1\to-2$.}
\(
L_{-}^4 Y_{2,2} = L_{-}\bigl(L_{-}^3 Y_{2,2}\bigr)
= -12\hbar^3\bigl(L_{-}Y_{2,-1}\bigr).
\)
Now,
\begin{align*}
L_{-}Y_{2,-1} &= -\,\hbar\,\sqrt{6 - (-1)(-2)}\,Y_{2,-2}
= -\hbar\,\sqrt{6-2}\,Y_{2,-2}
= -\,\hbar\cdot 2\,Y_{2,-2}
= -2\hbar\,Y_{2,-2}.
\end{align*}
Hence,
\(
L_{-}^4 Y_{2,2} = -12\hbar^3\bigl(-2\hbar\bigr)\,Y_{2,-2} = +24\hbar^4\,Y_{2,-2}.
\)

\subsection*{4. Conclusion and Normalization}
Each step precisely matches the standard factor:
\[
L_{-}Y_{l,m} = -\hbar\,\sqrt{l(l+1)-m(m-1)}\;Y_{l,m-1},
\]
and so we indeed obtain \(Y_{2,m}\) with correct normalization. In particular, after four applications of \(L_{-}\) to \(Y_{2,2}\), we get \(Y_{2,-2}\) up to the factor \(24\hbar^4\). This shows consistency with the general formula.








\section*{Question 2: Finding Normalized Eigenfunctions of $\hat{L}_x$ in the $\ell=1$ Space}

We know that the standard spherical harmonics $\{Y_{1,m}\}$ (with $m = -1,0,1$) form an orthonormal basis of eigenstates of $\hat{L}_z$. Now we want the linear combinations that diagonalize $\hat{L}_x$.

\subsection*{1. The $\ell=1$ Spherical Harmonics}
We have three states \($Y_{1,-1},\,Y_{1,0},\,Y_{1,1}\)$ which satisfy
\begin{align*}
\hat{L}_z Y_{1,m} &= m\hbar\,Y_{1,m}, \quad m\in\{-1,0,1\}.
\end{align*}
We now seek new linear combinations that diagonalize $\hat{L}_x$:
\begin{equation}
\hat{L}_x \Psi_{x,\alpha} = \alpha\hbar \Psi_{x,\alpha}, \quad \alpha\in\{-1,0,+1\}.
\end{equation}

\subsection*{2. Known Linear Combinations}
By explicit matrix diagonalization (or by analogy to spin-1 rotations), the orthonormal eigenstates of $\hat{L}_x$ in this subspace are:
\begin{align*}
\Psi_{x,+\hbar} &= \frac{1}{2}\Bigl(Y_{1,1} \; -\sqrt{2}\; Y_{1,0} \;+ Y_{1,-1}\Bigr),\\
\Psi_{x,0} &= \frac{1}{\sqrt{2}}\Bigl(Y_{1,1} \; - Y_{1,-1}\Bigr),\\
\Psi_{x,-\hbar} &= \frac{1}{2}\Bigl(Y_{1,1} \; +\sqrt{2}\; Y_{1,0} \;+ Y_{1,-1}\Bigr).
\end{align*}
These states each have eigenvalues \(\pm\hbar\) or 0 under $\hat{L}_x$, and they are all normalized due to the orthonormality of $Y_{1,m}$.

\subsection*{Answer (Question 2)}
\textbf{Eigenfunctions of $\hat{L}_x$ with $\ell=1$:}
\begin{itemize}
\item \(\hat{L}_x\)-eigenvalue $+\hbar$:\quad $\displaystyle \Psi_{x,+\hbar} = \frac{1}{2}\bigl(Y_{1,1} - \sqrt{2}\,Y_{1,0} + Y_{1,-1}\bigr).$
\item \(\hat{L}_x\)-eigenvalue $0$:\quad $\displaystyle \Psi_{x,0} = \frac{1}{\sqrt{2}}\bigl(Y_{1,1} - Y_{1,-1}\bigr).$
\item \(\hat{L}_x\)-eigenvalue $-\hbar$:\quad $\displaystyle \Psi_{x,-\hbar} = \frac{1}{2}\bigl(Y_{1,1} + \sqrt{2}\,Y_{1,0} + Y_{1,-1}\bigr).$
\end{itemize}







\section*{Question 3: Degeneracy of the 3D Isotropic Harmonic Oscillator in Spherical Coordinates}

We already know from Cartesian coordinates that the $n$-th excited state has degeneracy
\begin{equation}
\label{eq:cartesian-deg}
 g_n = \binom{n+2}{2} = \frac{(n+1)(n+2)}{2}.
\end{equation}
Here is how to see the same result in \emph{spherical} coordinates.

\subsection*{1. Spherical-Coordinate Labeling}
In spherical coordinates, each energy eigenstate of the 3D isotropic harmonic oscillator is labeled by \(n_r\ge 0\) (radial) and \(\ell\ge 0\) (angular momentum). The total energy level index is \(N = 2n_r + \ell\). Thus the energy is proportional to \(N + \tfrac32\). If we call \(n = N\) the excitation level, then
\begin{equation}
 n = 2n_r + \ell.
\end{equation}
For a given \(\ell\), there are \(2\ell + 1\) possible \(m\) values, i.e., the usual degeneracy from angular momentum.

\subsection*{2. Summation Over $\ell$}
We must sum over all integer pairs \((n_r,\ell)\) such that \(2n_r + \ell = n\). That is equivalent to summing over \(\ell = n, n-2, n-4, \dots\) down to 0 or 1, depending on parity of \(n\). Each \(\ell\) contributes an angular degeneracy of \(2\ell + 1\). Hence:
\begin{equation}
\label{eq:spherical-deg}
 g_n = \sum_{\substack{\ell=0\text{ (or 1)}\\\ell\equiv n\,\mathrm{mod}\,2}}^{\ell\le n} (2\ell + 1).
\end{equation}

\subsection*{3. Computing the Sum}
\paragraph{Case 1: $n=2p$ (Even).}\quad Then \(\ell = 2p, 2p-2, \dots, 0\). The number of terms is \(p+1\). Summation of \((2\ell + 1)\) over these values yields exactly the same binomial coefficient result:
\[
\sum_{k=0}^{p} [2(2p - 2k) + 1] = (n+1)(n+2)/2, \quad\text{where }n=2p.
\]
\paragraph{Case 2: $n=2p+1$ (Odd).}\quad Then \(\ell = 2p+1, 2p-1, \dots, 1\). A similar arithmetic series argument gives the same final formula:
\[
 g_n = \binom{n+2}{2} = \frac{(n+1)(n+2)}{2}.
\]

\subsection*{Conclusion}
Either by Cartesian counting (Eq.~\ref{eq:cartesian-deg}) or by summation in spherical coordinates (Eq.~\ref{eq:spherical-deg}), we get the \emph{same} degeneracy
\begin{equation}
\boxed{g_n = \frac{(n+1)(n+2)}{2},}
\end{equation}
confirming consistency of the two methods.







\section*{Question 4: Computing $\langle r^2 \rangle$ in Two Cases}
\subsection*{(a) First Excited State of the 3D Isotropic Harmonic Oscillator}
For a 3D isotropic HO, energy levels are labeled by $(n_r,\ell)$, with total $N = 2n_r + \ell$. The first excited state corresponds to $N=1$, i.e. $(n_r=0,\ell=1)$. A known standard result is:
\begin{equation}
\langle r^2 \rangle_{n_r,\ell} = \frac{\hbar}{m\omega}\Bigl(2n_r + \ell + \tfrac{3}{2}\Bigr).
\end{equation}
Hence, for $(n_r=0,\ell=1)$:
\begin{equation}
\langle r^2 \rangle = \frac{\hbar}{m\omega}\Bigl(1 + \tfrac{3}{2}\Bigr) = \frac{5\hbar}{2\,m\,\omega}.
\end{equation}

\subsection*{(b) Ground State of the Infinite Spherical Well}
Consider a spherical well of radius $a$, $V(r)=0$ for $r\le a$ and $\infty$ otherwise. The $\ell=0$ ground state has radial wavefunction:
\begin{equation}
\Psi_0(r,\theta,\phi) = \frac{1}{\sqrt{4\pi}}\,\frac{1}{r}\sqrt{\frac{2}{a}}\,\sin\Bigl(\frac{\pi r}{a}\Bigr), \quad 0\le r\le a.
\end{equation}
Thus,
\begin{equation}
\langle r^2 \rangle = \int_0^a \int_\Omega |\Psi_0(r,\theta,\phi)|^2 r^2 \,d^3x.
\end{equation}
After integrating over angles, the radial integral becomes:
\begin{equation}
\langle r^2 \rangle = \int_0^a \Bigl(\frac{2}{a}\sin^2\bigl(\tfrac{\pi r}{a}\bigr)\Bigr) r^2 \,dr.
\end{equation}
Define $I = \int_0^a r^2\sin^2(\tfrac{\pi r}{a})\,dr$, and use $\sin^2 x = \tfrac12[1 - \cos(2x)]$. Standard integrals yield:
\begin{equation}
I = \frac{a^3}{6} - \frac{a^3}{4\pi^2},
\end{equation}
so
\begin{equation}
\langle r^2 \rangle = \frac{2}{a}\,I = 2\,\frac{1}{a}\Bigl(\frac{a^3}{6} - \frac{a^3}{4\pi^2}\Bigr)
= a^2 \Bigl(\frac{1}{3} - \frac{1}{2\pi^2}\Bigr).
\end{equation}

\subsection*{Answer (Question 4)}
\begin{itemize}
\item \textbf{First Excited State of 3D HO:}\quad $\displaystyle \langle r^2 \rangle = \frac{5\hbar}{2\,m\,\omega}.$
\item \textbf{Ground State of Infinite Spherical Well (radius $a$):}\quad $\displaystyle \langle r^2 \rangle = a^2\Bigl(\tfrac{1}{3} - \tfrac{1}{2\pi^2}\Bigr).$
\end{itemize}







\section*{Question 5: Finite Spherical Well ($\ell=0$ Ground State)}
Consider the spherical potential
\begin{equation}
V(r) = \begin{cases}
-\,V_0, & 0 \le r \le a,\\
0, & r > a,
\end{cases}
\end{equation}
with $V_0 > 0$. We seek the $\ell=0$ ground state, so the radial wavefunction is $u(r)/r$ with $u(0)=0$.

\subsection*{1. Radial Schr\"odinger Equation}
For $\ell=0$:
\begin{align*}
-\frac{\hbar^2}{2m}\frac{d^2 u}{dr^2} + V(r)\,u(r) &= E\,u(r).
\end{align*}
\paragraph{(a) Inside ($0\le r\le a$, $V=-V_0$):}
\[
-\frac{\hbar^2}{2m}\frac{d^2 u}{dr^2} - V_0\,u(r) = E\,u(r)\;\Longrightarrow\; -\frac{\hbar^2}{2m}\frac{d^2 u}{dr^2} = (E+V_0)\,u(r).
\]
Define $k^2 = \tfrac{2m}{\hbar^2}(V_0 + E)$ and get $u_{\text{in}}(r) = A\sin(kr) + B\cos(kr)$. Regularity at $r=0$ forces $B=0$, so $u_{\text{in}}(r)=A\sin(kr).$

\paragraph{(b) Outside ($r > a$, $V=0$):}
\(
-\frac{\hbar^2}{2m}\frac{d^2 u}{dr^2} = E\,u(r).\)
With $E<0$ for a bound state, let $\kappa^2 = -\tfrac{2mE}{\hbar^2}>0$. Then the physical solution is $u_{\text{out}}(r)=C e^{-\kappa r}.$

\subsection*{2. Boundary Conditions at $r=a$}
1. Continuity of $u(r)$: $u_{\text{in}}(a) = u_{\text{out}}(a)$. i.e.
\[
A\sin(ka) = C e^{-\kappa a}.
\]
2. Continuity of $u'(r)$: $u_{\text{in}}'(a) = u_{\text{out}}'(a)$. Inside derivative: $A\,k\cos(ka).$ Outside derivative: $-\kappa\,C e^{-\kappa a}.$ So
\[
A\,k\cos(ka) = -\kappa\,C e^{-\kappa a}.
\]
Eliminate $C$ using the first condition, leading to
\(
A\,k\cos(ka) = -\kappa\Bigl(\frac{A\,\sin(ka)}{e^{-\kappa a}}\Bigr)e^{-\kappa a}\;\Longrightarrow\; k\cos(ka) = -\kappa\,\sin(ka).\)
Hence
\begin{equation}
\label{eq:transcend}
\frac{k}{\kappa} = -\tan(ka).
\end{equation}
\subsection*{3. Defining $k,\kappa$ in terms of $E$}
\begin{align*}
&k^2 = \frac{2m}{\hbar^2}\bigl(V_0+E\bigr), \quad E+V_0>0,\\
&\kappa^2 = -\frac{2mE}{\hbar^2},\quad E<0.\
\end{align*}
Equation (\ref{eq:transcend}) becomes:
\begin{equation}
\sqrt{\frac{V_0 + E}{-E}} = -\tan\bigl(a\,\sqrt{\tfrac{2m}{\hbar^2}(V_0 + E)}\bigr).
\end{equation}
One solves this numerically for $E<0$ to find the $\ell=0$ ground-state energy.

\subsection*{Answer (Question 5)}
\textbf{The ground state for $\ell=0$} is found by solving the transcendental condition:
\begin{equation}
\boxed{
\sqrt{\frac{V_0 + E}{-E}}\;=\;-\,\tan\Bigl(a\,\sqrt{\tfrac{2m}{\hbar^2}(V_0 + E)}\Bigr),
}
\end{equation}
which is fully analogous to the 1D finite square well but in spherical geometry.








\end{document}