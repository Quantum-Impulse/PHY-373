\documentclass[12pt]{article}
\usepackage{amsmath,amssymb,physics}
\usepackage{enumitem}
\usepackage[margin=1in]{geometry}

% --- Convenience spacing commands ---
\newcommand{\qs}{\bigskip\bigskip}   % “question spacing”
\newcommand{\vv}{\medskip}           % small vertical space

% --- Disable paragraph indentation, add vertical skip instead ---
\setlength{\parindent}{0pt}
\setlength{\parskip}{6pt}

\begin{document}

\begin{center}
    {\Large\bfseries PHYSICS 373 \\
    Midterm Examination 1}\\[2ex]
    Time allowed: \textbf{1 hour} \qquad (Calculator permitted)
\end{center}

%---------------------------------------------------------------
\newpage
{\large\bfseries 1.\; True / False}\qs

\emph{For each statement below, circle \textbf{true} or \textbf{false}.}\vv

\begin{enumerate}[label=\textbf{\alph*)}, leftmargin=1.2cm]
  \item A \textbf{unitary} operator has eigenvalues that are purely~imaginary.
        \hfill \textbf{[\,true\,]  [\,false\,]}
  \vv
  \item The ground–state wave‑function of \emph{any} 1‑D quantum system is an
        \emph{even} function of~$x$.
        \hfill \textbf{[\,true\,]  [\,false\,]}
  \vv
  \item Eigenfunctions of a Hermitian operator with different eigenvalues are orthogonal.
        \hfill \textbf{[\,true\,]  [\,false\,]}
  \vv
  \item $\displaystyle
        \psi(x)=\frac{1}{\sqrt{2L\cosh(x/L)}}$ is a valid $t=0$ wave‑function for the 1‑D harmonic oscillator.
        \hfill \textbf{[\,true\,]  [\,false\,]}
  \vv
  \item In the ground state of a 1‑D quantum system, the momentum uncertainty
        $\Delta p$ is~zero.
        \hfill \textbf{[\,true\,]  [\,false\,]}
\end{enumerate}

%---------------------------------------------------------------
\newpage
{\large\bfseries 2.\; Operator algebra}\qs

Let
\[
    \boxed{\hat O_1 = i\,e^{-\beta x}\dv{}{x}}, \qquad \beta>0,
\]
acting on the Hilbert space $\mathcal H = L^2(\mathbb R)$.

\vv
\begin{enumerate}[label=\textbf{\alph*)}, leftmargin=1.2cm]
  \item What is the adjoint operator $\hat O_1^{\dagger}$?\vv

  \item For what (possibly complex) values of $a$ is
        \[
            \hat O = \hat O_1 + a\,e^{-\beta x}
        \]
        Hermitian?\vv

  \item Compute $\displaystyle\expval{\hat O}$ in the \emph{ground state}
        of the 1‑D harmonic oscillator.
\end{enumerate}

%---------------------------------------------------------------
\newpage
{\large\bfseries 3.\; Two–state system}\qs

A certain 2‑state system is governed by the Hamiltonian
\[
   \hat H = E_0
   \begin{pmatrix}
      2 & 1+i\\
      * & 3
   \end{pmatrix}.
\]

\vv
\begin{enumerate}[label=\textbf{\alph*)}, leftmargin=1.2cm]
  \item What is the missing entry ($*$) in $\hat H$?\vv

  \item Consider the state $\displaystyle \ket{\psi} =
        \begin{pmatrix}0 \\ 1\end{pmatrix}$.
        What is the expectation value $\expval{H}$ and the uncertainty
        $\Delta H$ in this state?\vv

  \item When you \emph{actually perform} an energy measurement on the system,
        what \emph{possible values} can be obtained?\vv

  \item What is the probability that a measurement in the state
        $\ket{\psi}$ yields the value $E_0$?
\end{enumerate}

%---------------------------------------------------------------
\newpage
{\large\bfseries 4.\; Table of integrals (given)}\qs

\[
  \int_0^\pi \sin^{2}\theta \, d\theta = \frac{\pi}{2},
  \qquad
  \int_{-\infty}^{\infty} e^{-y^{2}/a^{2}}\,dy = a\sqrt{\pi},
  \qquad
  \int_{-\infty}^{\infty} y^{2} e^{-y^{2}/a^{2}}\,dy = \frac{a^{3}}{2}\sqrt{\pi}.
\]

\bigskip\bigskip
\begin{center}
    {\small\slshape End of exam\,—\,Good luck!}
\end{center}

\end{document}
